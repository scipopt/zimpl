\begin{description}
%
% zimpl.c
%
\item[101 Bad filename]\ \\
   The name given with the \code{-o} option is either missing, 
   a directory name, or starts with a dot.
\item[102 File write error]\ \\
   Some error occurred when writing to an output file. A description of 
   the error follows on the next line. For the meaning 
   consult your OS documentation.
\item[103 Output format not supported, using LP format]\ \\
   You tried to select another format then \code{lp}, \code{mps}, or \code{hum}.
\item[104 File open failed]\ \\
   Some error occurred when trying to open a file for writing. A description of 
   the error follows on the next line. For the meaning 
   consult your OS documentation.
%
% inst.c
%
\item[105 Duplicate constraint name ``xxx'']\ \\
   Two \code{subto} statements have the same name.
\item[106 Empty LHS, constraint trivially violated]\ \\
   One side of your constraint is empty and the other not equal to
   zero. Most frequently this happens, when a set to be summed up is empty.
\item[107 Range must be $l\leq x\leq u$, or $u \geq x\geq l$]\ \\
   If you specify a range you must have the same comparison operators
   on both sides.
\item[108 Empty Term with nonempty LHS/RHS, constraint trivially
   violated]\ \\
   The middle of your constraint is empty and either the left- or
   right-hand side of the range is not zero.
   This most frequently happens, when a set to be summed up is empty.
\item[109 LHS/RHS contradiction, constraint trivially violated]\ \\
   The lower side of your range is bigger than the upper side, e.g.
   $15\leq x\leq 2$. 
\item[110 Division by zero]\ \\
   You tried to divide by zero. This is not a good idea.
\item[111 Modulo by zero]\ \\
   You tried to compute a number modulo zero. This does not work well.
\item[112 Exponent value \code{xxx} is too big or not an integer]\ \\
   It is only allowed to raise a number to the power of integers. Also trying to
   raise a number to the power of more than two billion is 
   prohibited.\footnote{The behavior of this operation could 
   easily be implemented as \code{for(;;)} or more elaborate as 
   \code{void f()\{f();\}}.}
\item[113 Factorial value \code{xxx} is too big or not an integer]\ \\
   You can only compute the factorial of integers.
   Also computing the factorial of a number bigger then two billion
   is generally a bad idea. See also Error 115.
\item[114 Negative factorial value]\ \\
   To compute the factorial of a number it has to be positive.
   In case you need it for a negative number, remember that for all 
   even numbers the outcome will be positive and for all odd number negative.
\item[115 Timeout!]\ \\
   You tried to compute a number bigger than $1000!$. 
   See also the footnote to Error 112.
\item[116 Illegal value type in min: \code{xxx} only numbers are
   possible]\ \\
   You tried to build the minimum of some strings.
\item[117 Illegal value type in max: \code{xxx} only numbers are
   possible]\ \\
   You tried to build the maximum of some strings.
\item[118 Comparison of different types]\ \\
   You tried to compare apples with oranges, i.e, numbers with
   strings. Note that the use of an undefined parameter could also
   lead to this message.
\item[119 \code{xxx} of sets with different dimension]\ \\
   To apply Operation \code{xxx} (union, minus, intersection, symmetric
   difference) on two sets, 
   both must have the same dimension tuples,\ie
   the tuples must have the same number of components.
\item[120 Minus of incompatible sets]\ \\
   To apply Operation \code{xxx} (union, minus, intersection, symmetric
   difference) on two sets, 
   both must have tuples of the same type,\ie
   the components of the tuples must have the same type (number,
   string).
%\item[121]\ \\
%\item[122]\ \\
\item[123 ``from'' value \code{xxx} is too big or not an
   integer]\ \\
   To generate a set, the ``from'' number must be an integer with an
   absolute value of less than two billion.
\item[124 ``upto'' value \code{xxx} is too big or not an
   integer]\ \\
   To generate a set, the ``upto'' number must be an integer with an
   absolute value of less than two billion.
\item[125 ``step'' value \code{xxx} is too big or not an
   integer]\ \\
   To generate a set, the ``step'' number must be an integer with an
   absolute value of less than two billion.
\item[126 Zero ``step'' value in range]\ \\
   The given ``step'' value for the generation of a set is
   zero. So the ``upto'' value can never be reached. 
\item[127 Illegal value type in tuple: \code{xxx} only numbers are
   possible]\ \\
   The selection tuple in a call to the \code{proj} function can
   only contain numbers.
\item[128 Index value \code{xxx} in proj too big or not an integer]\ \\
   The value given in a selection tuple of a \code{proj} function is
   not an integer or bigger than two billion.
\item[129 Illegal index \code{xxx}, set has only dimension
   \code{yyy}]\ \\
   The index value given in a selection tuple is bigger than the
   dimension of the tuples in the set.
% Removed in 2.02
%\item[130 Duplicate index \code{xxx} for initialization]\ \\
%   In the initialization of a indexed set, two initialization elements
%   have the same index. 
%   E.g, \code{set A[] := $<$1$>$ \{ 1 \}, $<$1$>$ \{ 2 \};}
\item[131 Illegal element \code{xxx} for symbol]\ \\
   The index tuple used in the initialization list of a index set, is
   not member of the index set of the set.
   E.g, \code{set A[\{ 1 to 5 \}] := $<$1$>$ \{ 1 \}, $<$6$>$ \{ 2 \};}
\item[132 Values in parameter list missing, probably wrong read
  template]\ \\
  Probably the template of a read statement looks
  like \code{"$<$1n$>$"} only having a tuple, instead of \code{"$<$1n$>$ 2n"}.
\item[133 Unknown symbol \code{xxx}]\ \\
  A name was used, that is not defined anywhere in scope.
\item[134 Illegal element \code{xxx} for symbol]\ \\
  The index tuple given in the initialization is not member of the
  index set of the parameter.
\item[135 Index set for parameter \code{xxx} is empty]\ \\
  The attempt was made to declare an indexed parameter with the
  empty set as index set. Most likely the index set has a \code{with}
  clause which has rejected all elements.
%
% space here
%
\item[139 Lower bound for integral var \code{xxx} truncated to
  \code{yyy}] (warning)\ \\
  An integral variable can only have an integral bound. So the given
  non integral bound was adjusted.
\item[140 Upper bound for integral var \code{xxx} truncated to
  \code{yyy}] (warning)\ \\
  An integral variable can only have an integral bound. So the given
  non integral bound was adjusted.
\item[141 Infeasible due to conflicting bounds for var \code{xxx}]\ \\
  The upper bound given for a variable was smaller than the lower bound.
\item[142 Unknown index \code{xxx} for symbol \code{yyy}]\ \\
  The index tuple given is not member of the index set of the symbol.
\item[143 Size for subsets \code{xxx} is too big or not an integer]\ \\
  The cardinality for the subsets to generate must be given as an
  integer smaller than two billion.
\item[144 Tried to build subsets of empty set]\ \\
  The set given to build the subsets of, was the empty set.
\item[145 Illegal size for subsets \code{xxx}, should be between 1 
  and \code{yyy}]\ \\
  The cardinality for the subsets to generate must be between 1 
  and the cardinality of the base set.
\item[146 Tried to build powerset of empty set ]\ \\
  The set given to build the powerset of, was the empty set.
%
% iread.c
%
\item[147 use value \code{xxx} is too big or not an integer]\ \\
  The use value must be given as an integer smaller than two billion.
\item[148 use value \code{xxx} is not positive]\ \\
  Negative or zero values for the use parameter are not allowed.
\item[149 skip value \code{xxx} is too big or not an integer]\ \\
  The skip value must be given as an integer smaller than two billion.
\item[150 skip value \code{xxx} is not positive]\ \\
  Negative or zero values for the skip parameter are not allowed.
\item[151 Not a valid read template]\ \\
  A read template must look something like \code{"$<$1n,2n$>$"}.
  There have to be a $<$ and a $>$ in this order.
\item[152 Invalid read template syntax]\ \\
  Apart from any delimiters like \code{$<$}, \code{$>$}, and commas a
  template must consists of number character pairs like \code{1n}, \code{3s}.
\item[153 Invalid field number \code{xxx}]\ \\
  The field numbers in a template have to be between 1 and 255.
\item[154 Invalid field type \code{xxx}]\ \\
  The only possible field types are \code{n} and \code{s}.
\item[155 Invalid read template, not enough fields]\ \\
  There has to be at least one field inside the delimiters.
\item[156 Not enough fields in data]\ \\
  The template specified a field number that is higher than the actual
  number of field found in the data. 
\item[157 Not enough fields in data (value)]\ \\
  The template specified a field number that is higher than the actual
  number of field found in the data. The error occurred after the 
  index tuple in the value field.
\item[158 Read from file found no data]\ \\
  Not a single record could be read out of the data file. Either the
  file is empty, or all lines are comments.
%
% code.c
%  
\item[159 Type error, expected \code{xxx} got \code{yyy}]\ \\
  The type found was not the expected one, e.g. subtracting 
  a string from a number would result in this message.
%
% elem.c
% 
\item[160 Comparison of elements with different types \code{xxx} /
  \code{yyy}]\ \\
  Two elements from different tuples were compared and found to be 
  of different types. 
%
% load.c
%
\item[161 Line \code{xxx}: Unterminated string]\ \\
  This line has an odd number of \code{"} characters. 
  A String was started, but not ended.
\item[162 Line \code{xxx}: Trailing \code{"yyy"} ignored] (warning)\ \\
  Something was found after the last semicolon in the file.
\item[163 Line \code{xxx}: Syntax Error]\ \\
  A new statement was not started with one of the keywords:
  \code{set}, \code{param}, \code{var}, \code{minimize}, 
  \code{maximize}, \code{subto}, or \code{do}.
%
% set.c
%
\item[164 Duplicate element \code{xxx} for set rejected] (warning)\ \\
   An element was added to a set that was already in it.
\item[165 Comparison of different dimension sets] (warning)\ \\
   Two sets were compared, but have different dimension tuples.
   (This means they never had a chance to
   be equal, other then being empty sets.)
%
% symbol.c
%
\item[166 Duplicate element \code{xxx} for symbol \code{yyy} rejected]
   (warning)\ \\
  An element that was already there was added to a symbol.
%
% tuple.c
%
\item[167 Comparison of different dimension tuples] (warning)\ \\
  Two tuples with different dimensions were compared.
%
% zimpl.c
%
\item[168 No program statements to execute]\ \\
  No \zimpl statements were found in the files loaded.
%
% code.c
%
\item[169 Execute must return void element]\ \\
  This should not happen. If you encounter
  this error please email the \code{.zpl} file to \url{mailto:koch@zib.de}.
%
% inst.c
%
\item[170 Uninitialized local parameter \code{xxx} in call of
  define \code{yyy}]\ \\
  A define was called and one of the arguments was a ``name'' 
  (of a variable) for which no value was defined.
\item[171 Wrong number of arguments (\code{xxx} instead of \code{yyy})
  for call of define \code{zzz}]\ \\
  A define was called with a different number of arguments than in
  its definition.
\item[172 Wrong number of entries (\code{xxx}) in table line, 
  expected \code{yyy} entries]\ \\
  Each line of a parameter initialization table must have
  exactly the same number of entries as the index (first) line of
  the table.
\item[173 Illegal type in element \code{xxx} for symbol]\ \\
  A parameter can only have a single value type. Either numbers or
  strings. In the initialization both types were present.
%
% iread.c
%
\item[174 Numeric field \code{xxx} read as \code{"yyy"}. This is not a
  number]\ \\
  It was tried to read a field with an 'n' designation in the template,
  but what was read is not a valid number.
%
% zimpl.c
%
\item[175 Illegal syntax for command line define \code{"xxx"} --
  ignored] (warning)\\
  A parameter definition using the command line \code{-D} flag, must
  have the form \code{name=value}. The \code{name} must be a legal
  identifier, \ie it has to start with a letter and may consist only out
  of letters and numbers including the underscore. 
%
% vinst.c
%
\item[176 Empty LHS, in Boolean constraint] (warning)\ \\
   The left hand side, \ie the term with the variables, is empty. 
\item[177 Boolean constraint not all integer]\ \\
   No continuous (real) variables are allowed in a Boolean constraint.
\item[178 Conditional always true or false due to bounds] (warning)\ \\
   All or part of a Boolean constraint are always either true or
   false, due to the bounds of variables.
\item[179 Conditional only possible on bounded constraints]\ \\
   A Boolean constraint has at least one variable without finite bounds.
\item[180 Conditional constraint always true due to bounds] (warning)\ \\
   The result part of a conditional constraint is always true anyway. 
   This is due to the bounds of the variables involved.
\item[181 Empty LHS, not allowed in conditional constraint]\ \\
   The result part of a conditional constraint may not be empty.
\item[182 Empty LHS, in variable vabs]\ \\
   There are no variables in the argument to a \code{vabs} function.
   Either everything is zero, or just use \code{abs}.
\item[183 vabs term not all integer]\ \\
   There are non integer variables in the argument to a \code{vabs} function.
   Due to numerical reasons continuous variables are not allowed as
   arguments to \code{vabs}. 
\item[184 vabs term not bounded]\ \\
   The term inside a \code{vabs} has at least one unbounded variable.
\item[185 Term in Boolean constraint not bounded]\ \\
   The term inside a \code{vif} has at least one unbounded variable.
%
% inst.c
%
\item[186 Minimizing over empty set -- zero assumed] (warning)\ \\
   The index expression for the minimization was empty. The result
   used for this expression was zero.
\item[187 Maximizing over empty set -- zero assumed] (warning)\ \\
   The index expression for the maximization was empty. The result
   used for this expression was zero.
\item[188 Index tuple has wrong dimension]\ \\
   The number of elements in an index tuple is different from the
   dimension of the tuples in the set that is indexed.
\item[189 Tuple number \code{xxx} is too big or not an integer]\ \\
  The tuple number must be given as an integer smaller than two
  billion.
\item[190 Component number \code{xxx} is too big or not an integer]\ \\
  The component number must be given as an integer smaller than two
  billion.
\item[191 Tuple number \code{xxx} is not a valid value between 1..\code{yyy}]\ \\
  The tuple number must be between one and the cardinality of the set.
\item[192 Component number \code{xxx} is not a valid value between 1..\code{yyy}]\ \\
  The component number must be between one and the dimension of the set.
\item[193 Different dimension tuples in set initialization]\ \\
  The tuples that should be part of the list have different dimension.
\item[194 Indexing tuple \code{xxx} has wrong dimension
           \code{yyy}, expected \code{zzz}]\ \\
  The index tuple of an entry in a parameter initialization list must
  have the same dimension as the indexing set of the parameter. This
  is just another kind of error 134.
\item[195 Genuine empty set as index set]\ \\
  The set of an index set is always the empty set.
\item[196 Indexing tuple \code{xxx} has wrong dimension
           \code{yyy}, expected \code{zzz}]\ \\
  The index tuple of an entry in a set initialization list must
  have the same dimension as the indexing set of the set. 
  If you use a \code{powerset} or \code{subset} instruction, the index
  set has to be one dimension.
\item[197 Empty index set for set]\ \\
  The index set for a set is empty.
\item[198 Incompatible index tuple]\ \\
  The index tuple given had fixed components. The type of such a
  component was not the same as the type of the same component of tuples
  from the set.
\item[199 Constants are not allowed in SOS declarations]\ \\
  When declaring an SOS, weights are only allowed together with
  variabled. A weight alone does not make sense.
\item[200 Weights are not unique for SOS \code{xxx} (warning)]\ \\
  All weights assigned to variables in an special ordered set have to
  be unique. 
\item[201 Invalid read template, only one field allowed]\ \\
  When reading a single parameter value, the read template must
  consist of a single field specification.
\item[202 Indexing over empty set] (warning)\ \\
  The indexing set turns out to be empty.
\item[203 Indexing tuple is fixed] (warning)\ \\
  The indexing tuple of an index expression is completely fixed. As a
  result only this one element will be searched for.
\item[204 Randomfunction parameter minimum= \code{xxx} $>=$ maximum=
  \code{yyy}]\ \\
  The second parameter to the function \code{random} has to be
  strictly greater than the first parameter.
\item[205 \code{xxx} excess entries for symbol \code{yyy} ignored ]
  (warning)\ \\
  When reading the data for symbol \code{yyy} there were 
  \code{xxx} more entries in the file than indices for the symbol.
  The excess entries were ignored.  
\item[206 argmin/argmax over empty set] (warning)\ \\
   The index expression for the \code{argmin} or \code{argmax} was
   empty. The result is the empty set.
\item[207 ``size'' value \code{xxx} is too big or not an integer]\ \\
   The size argument for an \code{argmin} or \code{argmax} function
   must be an integer with an absolute value of less than two billion.
\item[208 ``size'' value \code{xxx} not >= 1]\ \\
   The size argument for an \code{argmin} or \code{argmax} function
   must be at least one, since it represents the maximum cardinality
   of the resulting set.
\item[209 MIN of set with more than one dimension]\ \\
   The expressions \code{min(A)} is only allowed if the elements of 
   set A consist of 1-tuples containing numbers.  
\item[210 MAX of set with more than one dimension]\ \\
   The expressions \code{max(A)} is only allowed if the elements of 
   set A consist of 1-tuples containing numbers.  
\item[211 MIN of set containing non number elements]\ \\
   The expressions \code{min(A)} is only allowed if the elements of 
   set A consist of 1-tuples containing numbers.  
\item[212 MAX of set containing non number elements]\ \\
   The expressions \code{max(A)} is only allowed if the elements of 
   set A consist of 1-tuples containing numbers.  
\item[213 More than 65535 input fields in line \code{xxx} of
   \code{yyy} (warning)]\ \\
   Input data beyond field number 65535 in line \code{xxx} of file
   \code{yyy} are ignored. Insert some newlines into your data!
\item[214 Wrong type of set elements -- wrong read template?]\ \\
   Most likely you have tried read in a set from a stream using
   \code{"n+"} instead of \code{"<n+>"} in the template. 
%
% space here
%
\item[216 Redefinition of parameter \code{xxx} ignored]\ \\
   A parameter was declared a second time with the same name. The
   typical use would be to declare default values for a parameter in
   the \zimpl file and override them by command-line defined.
\item[217 begin value \code{xxx} in substr too big or not an integer]\ \\
   The begin argument for an \code{substr} function
   must be an integer with an absolute value of less than two billion.
\item[218 length value \code{xxx} in substr too big or not an integer]\ \\
   The length argument for an \code{substr} function
   must be an integer with an absolute value of less than two billion.
\item[219 length value \code{xxx} in substr is negative]\ \\
   The length argument for an \code{substr} function
   must be greater or equal to zero.
%
% space here
%
\item[252 Startvals violate constraint, \ldots]\ \\
   If the given startvals are summed up, they violate the
   constraint. details about the sum of the LHS and the RHS are given
   in the warning message.
% 
% numbgmp.c
%
\item[700 log(): \code{OS specific domain or range error message}]\ \\
Function \code{log} was called with a zero or negative argument, or
the argument was too small to be represented as a \code{double}.
\item[701 sqrt(): \code{OS specific domain error message}]\ \\
Function \code{sqrt} was called with a negative argument.
\item[702 ln(): \code{OS specific domain or range error message}]\ \\
Function \code{ln} was called with a zero or negative argument, or
the argument was too small to be represented as a \code{double}.
%\item[]\ \\
%\item[]\ \\
%\item[]\ \\
%\item[]\ \\
%\item[]\ \\
%\item[]\ \\
%\item[]\ \\
% mmlscan.l
\item[800 parse error: expecting \code{xxx} (or \code{yyy})]\ \\
  Parsing error. What was found was not what was expected.
  The statement you entered is not valid.
\item[801 Parser failed]\ \\
  The parsing routine failed. This should not happen. If you encounter
  this error please email the \code{.zpl} file to \url{mailto:koch@zib.de}.
\item[802 Regular expression error]\ \\
  A regular expression given to the \code{match} parameter of a
  \code{read} statement, was not valid. See error messages for details.
\item[803 String too long \code{xxx} $>$ \code{yyy}]\ \\
  The program encountered a string which is larger then 1 GB. 
%
% inst.c
%
\item[900 Check failed!]\ \\
  A \code{check} instruction did not evaluate to true. 
\end{description}

