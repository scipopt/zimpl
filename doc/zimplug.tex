% $Id: zimplug.tex,v 1.3 2003/10/03 09:02:27 bzfkocht Exp $
%* * * * * * * * * * * * * * * * * * * * * * * * * * * * * * * * * * * * * * *
%*                                                                           *
%*   File....: zimplug.tex                                                   *
%*   Name....: Zimpl User Guide                                              *
%*   Author..: Thorsten Koch                                                 *
%*   Copyright (C) 2003 by Author, All rights reserved                       *
%*                                                                           *
%* * * * * * * * * * * * * * * * * * * * * * * * * * * * * * * * * * * * * * *
%
\section{Introduction}
A linear program (\lp) might look like this:
$$
\begin{array}{rll}
\min& 2 x + 3 y\\
\mbox{subject to}& x + y& \leq 6\\
&x,y&\ge 0\\
\end{array}
$$
The usual format to store the description of such a problem is \mps
invented by \ibm \cite{IBM1997} long ago. 
Nearly all available \lp and \mip solvers can read this format.
While \mps is a nice format to punch into a punch card and at least a
reasonable format to read for a computer it is quite unreadable
for humans. 

{
\small
\begin{verbatim}
   NAME        ex1.mps
   ROWS
    N  OBJECTIV          
    L  c1                
   COLUMNS
       x         OBJECTIV             2
       x         c1                   1
       y         OBJECTIV             3
       y         c1                   1
   RHS
       RHS       c1                   6
   ENDATA
\end{verbatim}
}
Another possibility is the \lpf format \cite{ILOG2002}, which is more
readable\footnote{
The \lpf format has also some idiosyncratic restrictions. For example
variables should not be named \code{e12} or the like. And it is not
possible to specify ranged constraints.}
but is only supported by a few solvers. 
{
\small
\begin{verbatim}
   Minimize
    cost:  +2 x +3 y
   Subject to
    c1:  +1 x +1 y <= 6
   End
\end{verbatim}
}
But since each coefficient of the matrix $A$ must be stated
explicitly it is also not a desirable choice to develop a mathematical
model.

\medskip
Now, with \zimpl it is possible to write this:
{\small
\begin{verbatim}
   var x;
   var y;
   minimize cost: 2 * x + 3 * y;
   subto c1: x + y <= 6;
\end{verbatim}
}
and have it automatically translated into \mps or \lpf format.
While this looks not much different from what is in the \lpf format,
the difference can be seen, if we use indexed variables.
Here is an example. This is the \lp 
$$
\begin{array}{rl}
\min& 2 x_1 + 3 x_2 + 1.5 x_3\\
\mbox{subject to}&\sum^3_{i=1} x_i \leq 6\\
&x_i\ge 0\\
\end{array}
$$
And this is how to tell it \zimpl
{
\small
\begin{verbatim}
set I      := { 1 to 3 };
param c[I] := <1> 2, <2> 3, <3> 1.5;
var   x[I] >= 0;
minimize value: sum <i> in I : c[i] * x[i];
subto     cons: sum <i> in I : x[i] <= 6;
\end{verbatim}
}

% -----------------------------------------------------------------------------
% --- Invocation
% -----------------------------------------------------------------------------
\section{Invocation}

To run \zimpl on a model given in the file \code{ex1.zpl} type the command
\begin{verbatim}
   zimpl ex1.zpl
\end{verbatim}

The general case is \code{zimpl} \code{[options]} $<$\code{input-files}$>$.

\medskip
It is possible to give more than one input file. They are read one
after the other as if they were all one big file.
If any error occurs while processing, \zimpl will print out an 
error message and abort. In case everything goes well, the results 
will be written into two or three files, depending on the options specified.

The first file is the problem generated from the model in either 
\lpf or \mps format with extension \code{.lp} or \code{.mps}.
The next one is the ``table'' file, which has the extension \code{.tbl}.
This file lists all variable and constraint names used in the model 
and there corresponding name in the problem file. 

The reason for this name translation is that the \mps format
can only handle names up to eight characters long. Also the the \lp format
restricts the length of names to 16 characters.

The third file is and optional \cplex branching order file.

\medskip
The following options are possible (only the first two are normally
of interest):

\medskip
\begin{tabular}{lp{115mm}}
%\hline
-t {\em format} & Selects the output format. Can be either \code{lp}
                  which is default, or \code{mps} or \code{hum} which
                  is only human readable.\\
-o {\em name}   & Sets the base-name for the output files.\\ 
                & Defaults to the name of the first input file striped 
                  of the path and
                  extension.\\ 
-F {\em filter} & The output is piped through a filter. A \%s in the
                  string is replaced by the output filename. For example 
                  \code{-p "gzip -c >\%s.gz"} would compress all the 
                  output files.\\
-n {\em cform}  & Select the format for the generation of constraint
                  names. Can be either {\em cm} which will number them 
                  $1\ldots n$ with a `c' in front. {\em cn} will use 
                  the name supplied in the \code{subto} statement and 
                  number them $1\ldots n$ within the statement. 
                  {\em cf} will use the name given with the \code{subto},
                  then a $1\ldots n$ number like in \emph{cm} and then 
                  append all the local variables from the forall statements.\\
-v {\em 1..5}   & Set the verbosity level. 0 is quiet, 1 is default,
                  2 is verbose, 3 is chatter, and 5 is debug.\\
%\hline
-b & Enables bison debugging output.\\
-f & Enables flex debugging output.\\
-h & Prints a help message.\\
-p & Does some presolve analysis to fix variables.\\
-r & Writes an \cplex \code{ord} branching order file.\\
%\hline
\end{tabular}

\medskip
A typical invocation is for example:
\begin{verbatim}
   zimpl -o hardone -t mps data.zpl model.zpl
\end{verbatim}
This reads the files \code{data.zpl} and \code{model.zpl} and
produces \code{hardone.mps} and \code{hardone.tbl}.

\medskip
{\bfseries If \mps-output is specified for a maximization problem,
the objective function will be inverted.}
% -----------------------------------------------------------------------------
% --- Format
% -----------------------------------------------------------------------------
\section{Format}

Each \zpl-file consists of six types of statements: 
\begin{itemize}
\item Sets
\item Parameters
\item Variables
\item Objective
\item Constraints
\item Function definitions
\end{itemize}

Each statement ends with a semicolon \code{;}. 
Everything from a number-sign \code{\#} to 
the end of the line is treated as a
comment and is ignored.

If a line starts with the word \code{include} followed by a filename in double
quotation marks, this file is read instead of this line.

\subsection{Expressions}
\zimpl works on the lowest level with two types of data: Strings and
numbers. 

Wherever a number or string is required, it is also possible to give a
parameter of the right value type. Usually expressions are
allowed instead of just a number or a string.
The precedence of operators should be the usual one, but
parenthesis can allways be used to specify the order of evaluation explicitly.
If in doubt use parenthesis to be save.

\subsubsection{Numeric expressions}
A number in \zimpl can be given in the usual format, \eg as 2, -6.5 or 5.234e-12. 
Numeric expressions consist of numbers, numeric valued parameters, and
any of the following operators and functions:

\begin{tabular}{lll}
\code{a${}^\wedge$b}, \code{a**b}&$a$ to the power of $b$   & $a^b$\\
\code{a+b}                       &addition                  & $a+b$\\
\code{a-b}                       &subtraction               & $a-b$\\
\code{a*b}                       &multiplication            & $a\cdot b$\\
\code{a/b}                       &division                  & $a/b$\\
\code{a mod b}                   &modulo                    & $a\mod b$\\
\code{a div b}                   &integer divison           & \\
\code{abs(a)}                    &absolute value            & $|a|$\\
\code{floor(a)}                  &round down                & $\lfloor a\rfloor$\\
\code{ceil(a)}                   &round up                  & $\lceil a\rceil$\\
\code{a!}                        &factorial                 & $a!$\\
\code{min(S)}                    &minimum of a set          &$\min_{s\in S}$\\
\code{max(S)}                    &maximum of a set          &$\max_{s\in S}$\\
\code{min(a,b,c,\ldots,n)}       &minimum of a list         &$\min (a,b,c,\ldots,n)$\\
\code{max(a,b,c,\ldots,n)}       &maximum of a list         &$\min (a,b,c,\ldots,n)$\\
\code{card(S)}                   &cardinality of a set      &$|S|$\\
\code{if a then b else c}        &conditional               &\\
\end{tabular}

\medskip
%With $\min$ and $\max$ it is possible to find the minimum/maximum
%member of an one dimensional set of numeric values.
%\code{card} gives the cardinality of a set.

The following functions are only computed with normal double precision
floating point arithmetic. So be careful:

\medskip
\begin{tabular}{lll}
\code{sqrt(a)}                   &square root               & $\sqrt a$\\
\code{log(a)}                    &logarithm to base 10      & $\log_{10}a$\\
\code{ln(a)}                     &natural logarithm         & $\ln a$\\
\code{exp(a)}                    &exponential function      & $e^a$\\
%\code{random(a,b)}               &random number in $\left[a,b\right]$&\\
\end{tabular}


\subsubsection{String expressions}
A string is delimited by double quotation marks \code{"},
\eg, \code{"Hallo"}. 

The following is either a numeric or a string expression, depending if
\emph{expression} is a string or a numeric expression.

\code{if} \emph{boolean-expression} \code{then}
\emph{expression} \code{else} \emph{expression} \code{end}.

At this time no further functions or operators
for strings are implemented.

\subsubsection{Boolean expressions}
These evaluate either to \emph{true} or \emph{false}. For numbers and
strings the
relational operators $<$, $<=$, $==$, $!=$, $>=$, and $>$ are
defined.
Combinations of boolean expressions with \code{and},
or \code{or}, and negation with \code{not} are possible.
The expression \emph{tuple} \code{in} \emph{set-expression} can be used to test set
membership of a tuple. 

\subsection{Sets}
Sets consist of tuples. The tuples in a sets are unordered, \ie, each tuple can only
be once in a set. Sets are delimited by braces, \code{\{} and \code{\}},
resp.
Tuples consist of components. The components are ordered. 
Each tuple of a specific set has the same number of components. 
The components are either numbers or strings. The type of the $n$-th
component of each tuple of a set must be the same. A tuple starts and ends with
$<$ and $>$, resp., \eg \code{$<$1,2,"x"$>$}. The components are separated by commas.
If tuples are one-dimensional, it is possible 
to omit the tuple delimiters in a list of elements, but then 
they must be omitted from all tuples of the set, \eg \code{\{1,2,3}\}.

Sets can be defined with the set statement. This consists of
the keyword \code{set}, the name of the set, an assignment operator
(\code{:=}) and a valid set expression.

Sets are referenced by use of an \emph{template} tuple, consisting
of placeholders that are replaced by the value of the components of
the respective tuple. For example a set $S$ consisting of two dimensional
tuples could be referenced by \code{<a,b> in S}. If any of the
placeholders are actual values, only those tuples will be extacted
that match these values. For example \code{<1,b> in S} will only get
those tuples whose first component is \code{1}. Please not that if
one of the placeholders is the name of an already defined parameter,
set or variable, it will be subtituted which will either result in an
error or an actual value.

\paragraph{Examples}
{\small 
\begin{verbatim}
set A := { 1, 2, 3 };
set B := { "hi", "ha", "ho" };
set C := { <1,2,"x">, <6,5,"y">, <787,12.6,"oh"> };
\end{verbatim}
}

For set expressions these functions and
operators are defined:

\medskip
\begin{tabular}{llp{7cm}}
%\hline
\code{A*B},&\\
\code{A cross B}   &Cross product & $\{(x,y)| x\in A\wedge y\in B\}$\\
\code{A+B},&\\
\code{A union B}   &Union & $\{x | x\in A\vee x\in B\}$\\
\code{A inter B}   &Intersection & $\{x |x\in A\wedge x\in B\}$\\
\code{A$\setminus$B, A-B},&\\
\code{A without B} &Difference & $\{ x |x\in A\wedge x\not\in B\}$\\
\code{A symdiff B} &Symmetric difference & 
   $\{ x |(x\in A\wedge x\not\in B)\vee(x\in B\wedge x\not\in A)\}$\\
\code{\{n\,{..}\,m\}},&\\
\code{\{n to m {\em by s}\}}& Generate, & 
   $\{ x | x=n + is \leq m, i\in\N_0\}$\\
   &(default $s = 1$)&\\ 
\code{proj(A, t)}& Projection & 
   The new set will consist of $n$-tuples, with\\
   &$t=(e_1,\ldots,e_n)$&
   the $i$-th component beeing the $e_i$-th component of $A$.\\
\code{if a then}\\
\code{\ b else c}        &conditional               &\\
%\hline
\end{tabular}

\medskip
An example for the use of the \code{if} \emph{boolean-expression} \code{then}
\emph{set expression} \code{else} \emph{set expression} \code{end} can
be found below with the examples for indexed sets.

\paragraph{Examples}
{\small 
\begin{verbatim}
set D := A cross B;
set E := { 6 to 9 } union A without { <2>, <3> }; 
set F:= { 1 to 9 } * { 10 to 19 } * { "A", "B" };
set G := proj(F, <3,1>) 
# will give: <"A",1>, <"A",2"> ... <"B",9>
\end{verbatim}
}

\subsubsection{Conditional sets}
It is possible to restrict a set to tuples that 
satisfy a boolean expression. The expression given by the \code{with}
clause is evaluated for each tuple in the set and only tuples for
which the expression evaluated to \emph{true} are included in the new set.

\paragraph{Examples}
{\small 
\begin{verbatim}
set F := { <i,j> in Q with i > j and i < 5 };
set A := { "a", "b", "c" };
set B := { 1, 2, 3 };
set V := { <a,2> in A*B with a == "a" or a == "b" };
# will give: <"a",2>, <"b",2>
\end{verbatim}
}


\subsubsection{Indexed sets}
It is possible to index one set with another set.
Here is a list of functions that use this:

\medskip
\begin{tabular}{llp{5cm}}
\code{powerset(A)}& Generates all subsets of $A$&$\{x | x\subseteq A\}$\\
\code{subset(A,n)}& Generates all subsets of $A$\\
                    & with $n$ elements&$\{x | x\subseteq A\wedge |x|=n\}$\\
\code{indexset(A)}&The index set of $A$&$\{1\ldots |A|\}$\\
\end{tabular}

\medskip
Indexed sets are accessed by adding the index of the set in brackets
\code{[} and \code{]}, like \code{S[7]}. 
There are three possibilites how to assign to an indexed set:
\begin{itemize}
\item The assignment expression is a list of comma separated pairs, 
      consisting of a tuple from the index set and a set expression to assign.
\item A set reference expression is given as index, then
      the assignment expression is evaluated for each index tuple.
\item By use of a function that returns a indexed set.
\end{itemize}

\subsubsection{Examples}
{\small 
\begin{verbatim}
set I           := { 1..3 };
set A[I]        := <1> { "a", "b" }, 
                   <2> { "c", "e" }, <3> { "f" };
set B[<i> in I] := { 3 * i };
set P[]         := powerset(N);
set J           := indexset(P);
set S[]         := subset(I, 2);
set K[<i> in I] := if i mod 2 == 0 then { i } else { -i } end;
\end{verbatim}
}

\subsection{Parameters}
Parameters can be declared with or without an indexing
set. Without indexing the parameter is just one value, which is either
a number or a string. For indexed parameters there is one
value for each member of the set. It is possible to declare a
\emph{default} value.

Parameters are declared in the following way: 
The keyword \code{param} is followed by the name of the parameter
optionally followed by the indexing set. 
Then after the assignment sign comes a list of pairs. The first element of each
pair is a tuple from the index set, the second element is the value of
the parameter for this index.

\subsubsection{Examples}
{\small 
\begin{verbatim}
param q := 5;
param u[A] := <1> 17, <2> 29, <3> 12 default 99;
param w[C] := <1,2,"x"> 1/2, <6,5,"y"> 2/3;
param x[<i> in I with i mod 2 == 0] := 3 * i;
\end{verbatim}
}

\medskip
In the example, no value is given for index $<$787,12.6,"oh"$>$ of
parameter \code{w}, that is assignments need not to be complete. This
is correct as long as it is never referenced.

\subsubsection{Parameter tables}
It is possible to initialise multi-dimensional indexed parameters from
tables. This is especially usefull for two-dimensional parameters.
The data is put in a table structure with \code{$|$} signs on each
margin. Then a head line with column indices has to be added, and one index
for each row of the table is needed. The row index can also be
multi-dimensional. The complete index for the entry is build by
appending the colum index to the row index.
The value entries are separated by commas. Any valid expression is
allowed here. As can be seen in the third example below, it is
possible to add a list of entries after the table.

\subsubsection{Examples}
{\small 
\begin{verbatim}
set I := { 1 .. 10 };
set J := { "a", "b", "c", "x", "y", "z" };

param h[I*J] :=   | "a", "c", "x", "z"   |
                |1|  12,  17, 99,     23 |
                |3|   4,   3,-17, 66*5.5 |
                |5| 2/3, -.4,  3, abs(-4)|
                |9|   1,   2,  0,      3 | default -99;

param g[I*I*I] :=     | 1, 2, 3 |
                  |1,1| 0, 1, 0 |
                  |1,2| 1, 1, 1 |
                  |1,3| 0, 0, 1 | 
                  |2,1| 1, 0, 1 |;

param k[I*I] :=  |  7,  8,  9 |
               |4| 89, 67, 55 |
               |5| 12, 13, 14 |, <1,2> 17, <3,4> 99;
\end{verbatim}
}

\subsection{Variables}
Like parameters, variables can be indexed. 
A variable has to be one out of three possible types: 
Continuous (called {\em real}), binary or integer. The default is real. 
Variables may have lower and upper bounds. Defaults are
zero as lower and infinity as upper bound. Binary variables are
always bounded between zero and one.
It is possible to compute the value of the lower or upper bounds
depending on the index for the variable (see last declaration in the
example). Bounds can also be set to \code{infinity} and \code{-infinity}.

\subsubsection{Examples}
{\small 
\begin{verbatim}
var x1;
var x2 binary;
var y[A] real >= 2 <= 18;
var z[<a,b,c> in C] integer 
    >= a * 10 
    <= if b <= 3 then p[b] else 10 end; 
\end{verbatim}
}

\fbox{Remember: if nothing is specified a lower bounds of zero is assumed.}

\subsection{Objective}
There must be at most one objective statement in a model. The objective
can be either \code{minimize} or \code{maximize}. Following the
keyword is a name, a colon (\code{:}) and then a term consisting 
of variables.

\subsubsection{Example}
{\small 
\begin{verbatim}
minimize cost: 12 * x1 -4.4 * x2 
   + sum <a> in A : u[a] * y[a]
   + sum <a,b,c> in C with a in E and b > 3 : -a/2 * z[a,b,c];
\end{verbatim}
}

\subsection{Constraints}
The general format for a constraint is 
\code{subto name: term sense term}.
Name can be any name starting with a letter. The term is defined
as in the objective. Sense is one of
\code{<=}, \code{>=} and \code{==}. 
Many constraints can be generated with one statement by the use of the
\code{forall} instruction, see below.

\subsubsection{Examples}
{\small 
\begin{verbatim}
subto time : 3 * x1 + 4 * x2 <= 7;
subto space: sum <a> in A : 2 * u[a] * y[a] >= 50;
subto weird: forall <a> in A : 
   sum <a,b,c> in C : z[a,b,c] == 55;
subto c21: 6 * (sum <i> in A : x[i] 
   + sum <j> in B : y[j]) >= 2;
subto c40: x[1] == a[1] + 
   2 * sum <i> in A do 2*a[i]*x[i]*3 + 4;
\end{verbatim}
}

\subsection{Details on \code{sum} and \code{forall}}
The general forms are

\qquad\code{forall} \emph{index} \code{do} \emph{term}
and 
\code{sum} \emph{index} \code{do} \emph{term}.
 
It is possible to nest several forall instructions.
%It should be noted, that a \code{sum}-expression is a \emph{term}
%itself, so it is possible to nest or concatenate them.
The general form of \emph{index} is 

\qquad\emph{tuple} \code{in} \emph{set} \code{with} \emph{boolean-expression}.

It is allowed to write a colon (\code{:}) instead of \code{do} and a
vertical bar (\code{|}) instead of \code{with}.
The number of components in the \emph{tuple} and in the components of the
mebers of the \emph{set} must match. The \code{with} part of an \emph{index} is
optional. The \emph{set} can be any expression giving a set.

\subsubsection{Examples}
{\small 
\begin{verbatim}
forall <i,j> in X cross { 1 to 5 } without { <2,3> } 
   with i > 5 and j < 2 do 
      sum <i,j,k> in X cross { 1 to 3 } cross Z do 
         p[i] * q[j] * w[j,k] >= if i == 2 then 17 else 53;
\end{verbatim}
}

Note that in the example \emph{i} and \emph{j} are set by the \code{forall}
instruction. So they are fixed for all invocations of \code{sum}.

\subsection{Details on \code{if} in constraints}
It is possible to put two variants of a constraint into an
\code{if}-statement. The same is true for \emph{terms}.
It is also possible to have a \code{forall} statement inside
an \code{if}.

\subsubsection{Examples}
{\small 
\begin{verbatim}
subto c1: forall <i> in I do
  if (i mod 2 == 0) then  3 * x[i] >= 4
                    else -2 * y[i] <= 3 end;
subto c2: sum <i> in I :
     if (i mod 2 == 0) then 3 * x[i] else -2 * y[i] end <= 3;
\end{verbatim}
}

\subsection{Initialising sets and parameters from a file}
It is possible to load the values for a set or a parameter from a
file. The syntax is

\smallskip
\code{read} \emph{filename} \code{as} \emph{template} 
[\code{skip} \emph{n}] [\code{use} \emph{n}] 
[\code{fs} \emph{s}] [\code{comment} \emph{s}]

\smallskip
\emph{filename} is the name of the file to read.

\emph{template} is a string with a template for the tuples to
generate. Each input line from the file is split in fields. The
splitting is done according to the following rules:
Whenever a space, tab, comma, semicolon or double colon is encountered
a new field is started. Text that is enclosed in double quotes is not
split, the quotes are allways removed. When a field is split all space
and tab charaters around the splitting are removed. If the split is
due to a comma, semicolon or double colon, each occurence of these
characters starts a new field. 

\subsubsection{Examples}
{\small 
All these lines have three fields:
\begin{verbatim}
Hallo;12;3    
Moin   7  2
"Hallo, Peter"; "Nice to meet you" 77
,,2
\end{verbatim}
}

For each component of the tuple the number of the field
to use for the value is given, followed by either a \code{n} if the
field should be interpreted as a number or \code{s} for a string. 
Have a look at the example, it is quite obvious how it works.
After the template some optional modifiers can be given. The order
does not matter. 

\code{skip} \emph{n} instructs to skip the first
\emph{n} lines of the file. 

\code{use} \emph{n} limits the number of
lines to use to \emph{n}. 

\code{comment} \emph{s} sets a list of characters that start
comments in the file. Each line is ended when any of the comment
characters is found.

When a file is read, empty lines are skiped and not counted for the
\code{use} clause. They are counted for the \code{skip} clause.
 
\subsubsection{Examples}
{\small 
\begin{verbatim}
set P := { read "nodes.txt" as "<1s>" };

nodes.txt:
   Hamburg   ->   <"Hamburg">
   M�nchen   ->   <"M�nchen">
   Berlin    ->   <"Berlin">

set Q := { read "blabla.txt" as "<1s,5n,2n>" skip 1 use 2 };

blabla.txt:
   Name;Nr;X;Y;No     ->   skip   
   Hamburg;12;x;y;7   ->   <"Hamburg",7,12>
   Bremen;4;x;y;5     ->   <"Bremen,5,4>
   Berlin;2;x;y;8     ->   skip
   
param cost[P] := read "cost.txt" as "<1s> 4n" comment "#";

cost.txt:
   # Name Price   ->   skip
   Hamburg 1000   ->   <"Hamburg"> 1000
   M�nchen 1200   ->   <"M�nchen"> 1200
   Berlin  1400   ->   <"Berlin">  1400

param cost[Q] := read "haha.txt" as "<3s,1n,2n> 4s";

haha.txt:
   1:2:ab:con1    ->   <"ab",1,2> "con1"
   2:3:bc:con2    ->   <"bc",2,3> "con1"
   4:5:de:con3    ->   <"de",4,5> "con1"
\end{verbatim}
}  

As with table format input, it is possible to add a list of tuples or
parameter entries after a read statement.
\subsubsection{Examples}
{\small 
\begin{verbatim}
set A := { read "test.txt" as "<2n>", <5>, <6> }; 
param winniepoh[X] := 
   read "values.txt" as "<1n,2n> 3n", <1,2> 17, <3,4> 29;
\end{verbatim}
}  


\subsection{Function Definitions}

It is possible to define function within \zimpl. The value a
function returns has to be either a number, a string or a set.
The arguments to a function can only be numbers or strings, but within
the function definition it is possible to access all otherwise
declared sets, parameters and variables.

Depending on the return value, the definition of a function has to start with 
\code{defnumb}, \code{defstrg} or \code{defset}. Then follows the name
of the function and a list of argument names enclosed in parenthesis.

After this comes an assignment operator (\code{:=}) and a valid
expression or set expression.

\subsubsection{Examples}
{\small 
\begin{verbatim}
defnumb dist(a,b)  := a*a + b*b;
defstrg huehott(a) := if a < 0 then "hue" else "hott" end;
defset  bigger(i)  := { <j> in K with j > i };
\end{verbatim}
}

\newpage
\section{Examples}
In this section we will show examples how to translate a problem
into \zimpl format.

\subsection{Diet problem}
This is the first example in 
\cite[Chapter 1, page 3]{Chvatal1983}.
It is a classic so-called \emph{diet}-problem.
Given a set of foods $F$ and a set of nutrients $N$, we have a table
$\pi_{fn}$ of the amount of nutrient $n$ in food $f$. Now $\Pi_n$
defines how much intake of each nutriement is needed. And $\Delta_f$
describes the maximum number of servings of each food. 

Now given prices $c_f$ for each food, we to find a selection of foods
that obeys the restrictions and has minimal cost. 
\begin{eqnarray}
\min_{f\in F, n\in N} c_{fn} x_{fn}&&\mbox{subject to}\\
\sum_{f\in F} \pi_f x_{fn} &\ge&\Pi_n\quad\forall n\in N\\
0\leq x_{fn}&\leq&\Delta_f\quad\forall f\in F, n\in N\\
x_{fn}&\in&\N\label{eqn:integrality}
\end{eqnarray}

As (\ref{eqn:integrality}) implies, only complete servings can be
optained. Half an egg is not an option.
Now translating this to \zimpl looks as follows:
{\small
\begin{verbatim}
set Food := {"Oatmeal","Chicken","Eggs","Milk","Pie","Pork"};
set Nutrients := { "Energy", "Protein", "Calcium" };
set Attr      := Nutrients + { "Servings", "Price" };

param needed[Nutrients] := 
  <"Energy"> 2000, <"Protein"> 55, <"Calcium"> 800;

param data[Food * Attr] := 
          |"Servings","Energy","Protein","Calcium","Price"|
|"Oatmeal"|        4 ,    110 ,       4 ,       2 ,     3 |
|"Chicken"|        3 ,    205 ,      32 ,      12 ,    24 |
|"Eggs"   |        2 ,    160 ,      13 ,      54 ,    13 |
|"Milk"   |        8 ,    160 ,       8 ,     284 ,     9 |
|"Pie"    |        2 ,    420 ,       4 ,      22 ,    20 |
|"Pork"   |        2 ,    260 ,      14 ,      80 ,    19 |;
#                       (kcal)       (g)      (mg) (cents)       
var x[<f> in Food] integer >= 0 <= data[f, "Servings"];

minimize cost: sum <f> in Food : data[f, "Price"] * x[f];

subto need :
  forall <n> in Nutrients do
    sum <f> in Food : data[f, n] * x[f] >= needed[n];
\end{verbatim}
}
\newpage
\subsection{TSP}
In this example we show how to genrate an exponential 
description of the \emph{Traveling Salesmen Problem} 
as given for example in 
\cite[Section 58.5]{Schrijver2003}.
The data is read in from a file giving the number of the city and the
x and y coordinate. Distances between cities are geometric.
A suitable data file would look like this:
{\small
\begin{verbatim}
#Ort         x y 
"Sylt"       1 1
"Flensburg"  3 1
"Neum�nster" 2 2
"Husum"      1 3
"Schleswig"  3 3
"Ausacker"   2 4
\end{verbatim}
}
The formulation in \zimpl follows below. Please note that \code{P[]}
holds all subsets of the cities. So don't try to solve a 52
city TSP this way. It won't work.
{\small
\begin{verbatim}
set V   := { read "tsp.dat" as "<1s>" comment "#" };
set E   := { <i,j> in V * V with i < j };
set P[] := powerset(V);
set K   := indexset(P);

param px[V] := read "tsp.dat" as "<1s> 2n" comment "#";
param py[V] := read "tsp.dat" as "<1s> 3n" comment "#";

defnumb dist(a,b) := sqrt((px[a]-px[b])^2 + (py[a]-py[b])^2);

var x[E] binary;

minimize cost: sum <i,j> in E : dist(i,j) * x[i, j];

subto two_connected:
   forall <v> in V do
      (sum <v,j> in E : x[v,j]) 
    + (sum <i,v> in E : x[i,v]) == 2;

subto no_subtour:
   forall <k> in K with 
      card(P[k]) > 2 and card(P[k]) < card(V) - 2 do
      sum <i,j> in E with <i> in P[k] and <j> in P[k] : x[i,j] 
      <= card(P[k]) - 1;
\end{verbatim}
}
\newpage
\subsection{Capacitated Facility Location Problem}
Here we have a formulation for the \emph{Capacitated Facility
Location} Problem. Of course this is also kind of a \emph{bin packing} problem
with packing costs and variable sized bins, or a \emph{cutting stock} problem
with cutting costs.

Given a set of possible plants $P$ to build, and a set of stores $S$
with a certain demand $\delta_s$ that has to be satisfied, we have
to decide which plant should serve which store.
We have costs $c_p$ for building plant $p$ and $c_{ps}$
for transporting the goods from plant $p$ to store $s$.
Each plant has only a limited capacity $\kappa_p$.
And we insist that each store is served by exactly one plant.
Of course we are looking for the cheapest solution:

\begin{eqnarray}
\min_{p\in P, s\in S} c_p + c_{ps}&&\mbox{subject to}\\ 
\sum_{p\in P} x_{ps}& =& 1\quad\forall s\in S\label{eqn:assign}\\
x_{ps}&\leq&z_{s}\quad\forall s\in S, p\in P\label{eqn:open}\\
\sum_{s\in S} \delta_s x_{ps}&\leq&\kappa_p\quad\forall p\in P\label{eqn:capacity}\\
x_{ps},z_p&\in&\BB
\end{eqnarray}

We have binary variables $z_p$, which are set to one, iff plant $p$ is
to be build. And we have binary variables $x_{ps}$, 
which are one iff plant $p$ serves shop $s$. 
Equation (\ref{eqn:assign}) demands that each store is assigned to
exactly one plant. Equation (\ref{eqn:open}) makes sure that a plant
that serves a shop is build. And Equaltion (\ref{eqn:capacity})
restricts the shops that are served by a plant to the plants capacity.
Putting this into \zimpl yields:
{\small
\begin{verbatim}
set PLANTS := { "A", "B", "C", "D" };
set STORES := { 1 .. 9 };
set PS     := PLANTS * STORES;

# How much does it cost to build a plant and what capacity 
# will it then have?
param building[PLANTS]:= <"A">500,<"B">600,<"C">700,<"D">800;
param capacity[PLANTS]:= <"A"> 40,<"B"> 55,<"C"> 73,<"D"> 90;

# Here is the demand of each store
param demand  [STORES]:= <1> 10,<2> 14,<3> 17,<4> 8,<5> 9,
                         <6> 12,<7> 11,<8> 15,<9> 16;

# Transportation cost from each plant to each store
param transport[PS] := 
      |  1,  2,  3,  4,  5,  6,  7,  8,  9 |
  |"A"| 55,  4, 17, 33, 47, 98, 19, 10,  6 |
  |"B"| 42, 12,  4, 23, 16, 78, 47,  9, 82 | 
  |"C"| 17, 34, 65, 25,  7, 67, 45, 13, 54 |
  |"D"| 60,  8, 79, 24, 28, 19, 62, 18, 45 |;

var x[PS]     binary;  # Is plant p suppling store s ?
var z[PLANTS] binary;  # Is plant p build ?

# We want it cheap
minimize cost: sum <p> in PLANTS : building[p] * z[p]
             + sum <p,s> in PS : transport[p,s] * x[p,s];

# Each store is supplied by exactly one plant
subto assign: 
  forall <s> in STORES : sum <p> in PLANTS : x[p,s] == 1;
   
# To be able to supply a store, a plant must be build
subto build: forall <p,s> in PS : x[p,s] <= z[p];

# The plant must be able to meet the demands from all stores
# that are connected to it
subto limit: forall <p> in PLANTS :
   sum <s> in S : demand[s] * x[p,s] <= capacity[p];
\end{verbatim}
}

The optimal solution in this case is to build plants \code{A} and
\code{C}. Stores 2, 3, and 4 are served by plant \code{A}, the rest by
plant \code{C}. Total cost would be 1457.

\section{Error messages}
Here is a list of the incomprehensible error messages \zimpl
can produce:

\begin{description}
%
% zimpl.c
%
\item[101 Bad filename]\ \\
   The name given with the \code{-o} option is either missing, 
   a directory name, or starts with a dot.
\item[102 File write error]\ \\
   Some error occured when writing to an output file. A description of 
   the error follows on the next line. For the meaning 
   consult your OS documentation.
\item[103 Output format not supported, using LP format]\ \\
   You tried to select another format then \code{lp}, \code{mps}, or \code{hum}.
\item[104 File open failed]\ \\
   Some error occured when trying to open a file for writing. A description of 
   the error follows on the next line. For the meaning 
   consult your OS documentation.
%
% inst.c
%
\item[105 Dublicate constraint name ``xxx'']\ \\
   Two \code{subto} statements have the same name.
\item[106 Empty LHS, constraint trivially violated]\ \\
   One side of your constraint is empty and the other not equal to
   zero. This mostly happens if a set to sum up is empty.
\item[107 Range must be $l\leq x\leq u$, or $u \geq x\geq l$]\ \\
   If you specify a range you must have the same comparison operators
   on both sides.
\item[108 Empty Term with nonempty LHS/RHS, constraint trivially
   violated]\ \\
   The middle of your constraint is empty and either the left- or
   right-hand side of the range is not zero.
   This mostly happens if a set to sum up is empty.
\item[109 LHS/RHS contradiction, constraint trivially violated]\ \\
   The lower side of your range is bigger than the upper side, e.g.
   $15\leq x\leq 2$. 
\item[110 Division by zero]\ \\
   You tried to divide by zero. This is not a good idea.
\item[111 Modulo by zero]\ \\
   You tried to compute a number modulo zero. This does not work well.
\item[112 Exponent value \code{xxx} is too big or not an integer]\ \\
   It is only allowed to raise a number to the power of integers. Also trying to
   raise a number to the power of more than two billion is 
   prohibited.\footnote{The behaviour of this operation could 
   easily be imlemented as \code{for(;;)} or more elaborate as 
   \code{void f()\{f();\}}.}
\item[113 Factorial value \code{xxx} is too big or not an integer]\ \\
   You can only compute the factorial of integers.
   Also computing the factorial of a number bigger then two billion
   is generaly a bad idea. See also Error 115.
\item[114 Negative factorial value]\ \\
   To compute the factorial of a number it has to be positive.
   In case you need it from a negative number, remember that for all 
   even numbers the outcome will be positive and for all odd number negative.
\item[115 Timeout!]\ \\
   You tried to compute a number bigger than $1000!$. 
   See also the footnote to Error 112.
\item[116 Illegal value type in min: \code{xxx} only numbers are
   possible]\ \\
   You tried to build the minimum of some strings.
\item[117 Illegal value type in max: \code{xxx} only numbers are
   possible]\ \\
   You tried to build the maximum of some strings.
\item[118 Comparison of different types]\ \\
   You tried to compare apples with oranges, i.e, numbers with
   strings. Note that the use of an undefined parmeter could also
   lead to this message.
\item[119 Union of incompatible sets]\ \\
   To unite two sets, both must have the same dimension tuples,\ie
   the tuples must have the same number of components.
\item[120 Minus of incompatible sets]\ \\
   To subtract two sets, both must have the same dimension tuples.
\item[121 Intersection of incompatible sets]\ \\
   To intersect two sets, both must have the same dimension tuples.
\item[122 Symetric Difference of incompatible sets]\ \\
   Two build the symetric difference of two sets, 
   both must have the same dimension tuples.
\item[123 ``from'' value \code{xxx} in range too big or not an
   integer]\ \\
   To generate a set, the ``from'' number must be an integer with an
   absolute value of less than two billion.
\item[124 ``upto'' value \code{xxx} in range too big or not an
   integer]\ \\
   To generate a set, the ``upto'' number must be an integer with an
   absolute value of less than two billion.
\item[125 ``step'' value \code{xxx} in range too big or not an
   integer]\ \\
   To generate a set, the ``step'' number must be an integer with an
   absolute value of less than two billion.
\item[126 Zero ``step'' value in range]\ \\
   The given ``step'' value for the generation of a set is
   zero. So the ``upto'' value can never be reached. 
\item[127 Illegal value type in tuple: \code{xxx} only numbers are
   possible]\ \\
   The selection typle in a call to the \code{proj} function can
   only contain numbers.
\item[128 Index value \code{xxx} in proj too big or not an integer]\ \\
   The value given in a selection tuple of a \code{proj} function is
   not an integer or bigger than two billion.
\item[129 Illegal index \code{xxx}, set has only dimension
   \code{yyy}]\ \\
   The index value given in a selection tuple is bigger than the
   dimension of the tuples in the set.
\item[130 Duplicate index \code{xxx} for initialisation]\ \\
   In the initialisation of a indexed set, two initialisation elements
   have the same index. 
   E.g, \code{set A[] := $<$1$>$ \{ 1 \}, $<$1$>$ \{ 2 \};}
\item[131 Illegal element \code{xxx} for symbol]\ \\
   The index tuple used in the initialisation list of a index set, is
   not member of the index set of the set.
   E.g, \code{set A[\{ 1 to 5 \}] := $<$1$>$ \{ 1 \}, $<$6$>$ \{ 2 \};}
\item[132 Values in parameter list missing, probably wrong read
  template]\ \\
  Probably the template of a read statement looks
  like \code{"$<$1n$>$"} only having a tuple, instead of \code{"$<$1n$>$ 2n"}.
\item[133 Unknown local symbol \code{xxx}]\ \\
  A (local) symbol was used, that is not defined anywhere in scope.
\item[134 Illegal element \code{xxx} for symbol]\ \\
  The index tuple given in the initialisation is not member of the
  index set of the parameter.
\item[135 Index set for parameter \code{xxx} is empty]\ \\
  The attempt was made to declare an indexed parameter with the
  empty set as index set. Most likely the index set has a \code{with}
  clause which has rejected all elements.
\item[136 Lower bound for var \code{xxx} set to infinity -- ignored]\ \\
  In the zimpl code something like \code{$\ge$ infinity} must have
  been appeared. This makes no sense and is therefore ignored.
\item[137 Upper bound for var \code{xxx} set to -infinity -- ignored]\ \\
  In the zimpl code something like \code{$\le$ -infinity} must have 
  been appeared. This makes no sense and is therefore ignored.
\item[138 Priority/Startval for continous var \code{xxx} ignored]\ \\
  There has been set a priority or a starting value for a continous 
  (\code{real}) variable. This is not possible and therefore ignored.
\item[139 Lower bound for integral var \code{xxx} truncated to
  \code{yyy}]\ \\
  An integral variable can only have an integral bound. So the given
  non integral bound was ajusted.
\item[140 Upper bound for integral var \code{xxx} truncated to
  \code{yyy}]\ \\
  An integral variable can only have an integral bound. So the given
  non integral bound was ajusted.
\item[141 Infeasible due to conflicting bounds for var \code{xxx}]\ \\
  The upper bound given for a variable was smaller than the lower bound.
\item[142 Unknown index \code{xxx} for symbol \code{yyy}]\ \\
  The index tuple given is not member of the index set of the symbol.
\item[143 Size for subsets \code{xxx} is too big or not an integer]\ \\
  The cardinality for the subsets to generate must be given as an
  integer smaller than two billion.
\item[144 Tried to build subsets of empty set]\ \\
  The set given to build the subsets of, was the empty set.
\item[145 Illegal size for subsets \code{xxx}, should be between 1 
  and \code{yyy}]\ \\
  The cardinality for the subsets to generate must be between 1 
  and the cardinality of the base set.
\item[146 Tried to build powerset of empty set ]\ \\
  The set given to build the powerset of, was the empty set.
%
% iread.c
%
\item[147 use value \code{xxx} is too big or not an integer]\ \\
  The use value must be given as an integer smaller than two billion.
\item[148 use value \code{xxx} is not positive]\ \\
  Negative or zero values for the use parameter are not allowed.
\item[149 skip value \code{xxx} is too big or not an integer]\ \\
  The skip value must be given as an integer smaller than two billion.
\item[150 skip value \code{xxx} is not positive]\ \\
  Negative or zero values for the skip parameter are not allowed.
\item[151 Not a valid read template]\ \\
  A read template must look something like \code{"$<$1n,2n$>$"}.
  There have to be a $<$ and a $>$ in this order.
\item[152 Invalid read template syntax]\ \\
  Apart from any delimiters like \code{$<$}, \code{$>$}, and commas a
  template must consists of number charater pairs like \code{1n}, \code{3s}.
\item[153 Invalid field number \code{xxx}]\ \\
  The field numbers in a template have to be between 1 and 255.
\item[154 Invalid field type \code{xxx}]\ \\
  The only possible field types are \code{n} and \code{s}.
\item[155 Invalid read template, not enough fields]\ \\
  There has to be at least one field inside the delimters.
\item[156 Not enough fields in data]\ \\
  The template specified a field number that is higher than the actual
  number of field found in the data. 
\item[157 Not enough fields in data (value)]\ \\
  The template specified a field number that is higher than the actual
  number of field found in the data. The error occured behind the 
  index tuple in the value field.
\item[158 Read from file found no data]\ \\
  Not a single record could be read out of the data file. Either the
  file is empty, or all lines are comments.
%
% code.c
%  
\item[159 Type error, expected \code{xxx} got \code{yyy}]\ \\
  The type found was not the expected one, e.g. subtracting 
  a string from a number would result in this message.
%
% elem.c
% 
\item[160 Comparison of elements with different types \code{xxx} /
  \code{yyy}]\ \\
  Two elements from different tuples were compared and found to be 
  of different types. Probably you have a index expression with a 
  constant element, that does not match the elements of the set.
  Example: \code{set B:=\{$<$i,"a"$>$ in \{$<$1,2$>$\}\};}
%
% load.c
%
\item[161 Line \code{xxx}: Unterminated string]\ \\
  This line has an odd number of \code{"} characters. 
  A String was started, but not ended.
\item[162 Line \code{xxx}: Trailing \code{"yyy"} ignored]\ \\
  Something was found after the last semicolon in the file.
\item[163 Line \code{xxx}: Syntax Error]\ \\
  Either a new statement was not started with one of the keywords:
  \code{set}, \code{param}, \code{var}, \code{minimize}, 
  \code{maximize}, \code{subto}, and \code{print}, or no name
  was given after the keyword.
%
% set.c
%
\item[164 Dublicate element \code{xxx} for set rejected]\ \\
   An element was added to a set that was already in it.
\item[165 Comparison of differen dimension sets]\ \\
   Two sets were compared, but have different dimension tuples.
   (This means they never had a chance other then beeing empty sets to
   be equal.)
%
% symbol.c
%
\item[166 Dublicate element \code{xxx} for symbol \code{yyy} rejected]\ \\
  An element that was already there was added to a symbol.
%
% tuple.c
%
\item[167 Comparison of different dimension tuples]\ \\
  Two tuples with different dimensions were compared.
  Probably you have a index expression which does not match the  
  dimension of the tuples in the set.
  Example: \code{set B:=\{$<$1$>$ in \{$<$1,2$>$\}\};}
%
% zimpl.c
%
\item[168 No program statements to execute]\ \\
  No \zimpl statements were found in the files loaded.
%
% code.c
%
\item[169 Execute must return void element]\ \\
  This should not happen. If you encounter
  this error please email the \code{.zpl} file to \url{mailto:koch@zib.de}.
%
% inst.c
%
\item[170 Use of unintialised local variable \code{xxx} in call of
  define \code{yyy}]\ \\
  A define was called and one of the arguments was a ``name'' 
  (of a variable) for which no value was defined.
\item[171 Wrong number of arguments (\code{xxx} instead of \code{yyy})
  for call of define \code{zzz}]\ \\
  A define was called with the a different number of arguments than in
  its definition.
\item[172 Wrong number of entries (\code{xxx}) in table line, 
  expected \code{yyy} entries]\ \\
  Each line of a parameter initialisation table must have
  exactly the same number of entries as the index (first) line of
  the table.
\item[173 Illegal type in element \code{xxx} for symbol]\ \\
  A parameter can be only have one type of values. Either numbers or
  strings. In the initialisation both types were present.
%
% iread.c
\item[174 Numeric field \code{xxx} read as \code{"yyy"}. This is not a
  number]\ \\
  It was tried to read a field with a 'n' designation in the template,
  but what was read is not a valid number.
%
% numbgmp.c
%
\item[700 log(): \code{OS specific domain or range error message}]\ \\
Function \code{log} was called with a zero or negative argument, or
the argument was too small to be represented as a \code{double}.
\item[701 sqrt(): \code{OS specific domain error message}]\ \\
Function \code{sqrt} was called with a negative argument.
\item[702 ln(): \code{OS specific domain or range error message}]\ \\
Function \code{ln} was called with a zero or negative argument, or
the argument was too small to be represented as a \code{double}.
%\item[]\ \\
%\item[]\ \\
%\item[]\ \\
%\item[]\ \\
%\item[]\ \\
% mmlscan.l
\item[800 parse error: expecting \code{xxx} (or \code{yyy})]\ \\
  Parsing error. What was found was not what was expected.
  The statement you entered is not valid.
\item[801 Parser failed]\ \\
  The parsing routine failed. This should not happen. If you encounter
  this error please email the \code{.zpl} file to \url{mailto:koch@zib.de}.
\end{description}
