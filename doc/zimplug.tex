% TODO: Bei den SET examples koennte man parallel das
%       Ergenis zeigen:
%       set A := { 1..4}     |  { 1, 2, 3, 4 }
%
%       min/max expr beispiel
%
\section{Introduction}
Consider the following linear program:
$$
\begin{array}{rll}
\min& 2 x + 3 y\\
\mbox{subject to}& x + y& \leq 6\\
&x,y&\ge 0\\
\end{array}
$$
The standard format used to feed such a problem into a solver 
is called \mps.
\ibm invented it for the Mathematical
Programming System/360 \cite{Kallrath2004b,Spielberg2004} in the sixties.
Nearly all available \lp and \mip solvers can read this format.
While \mps is a nice format to punch into a punch card and at least a
reasonable format to read for a computer, it is quite unreadable
for humans. For instance, the \mps file of the above linear program
looks as follows:

\medskip
\lstset{language=mps,%
basicstyle=\sffamily\footnotesize,%
numberstyle=\sffamily\tiny\color{siennabrown},stepnumber=1}
\begin{lstlisting}[frame=]{}
    NAME        ex1.mps
    ROWS
     N  OBJECTIV          
     L  c1                
    COLUMNS
        x         OBJECTIV             2
        x         c1                   1
        y         OBJECTIV             3
        y         c1                   1
    RHS
        RHS       c1                   6
    BOUNDS
     LO BND       x                    0
     LO BND       y                    0
    ENDATA
\end{lstlisting}

\bigskip
\noindent Another possibility is the \lpf format \cite{CPlex80}, which is more
readable\footnote{
The \lpf format has also some idiosyncratic restrictions. For example
variables should not be named \code{e12} or the like. And it is not
possible to specify ranged constraints.}
but is only supported by a few solvers. 
{
\small
\begin{verbatim}
   Minimize
    cost:  +2 x +3 y
   Subject to
    c1:  +1 x +1 y <= 6
   End
\end{verbatim}
}
\noindent But since each coefficient of the matrix $A$ must be stated
explicitly it is also not a desirable choice to develop a mathematical
model.

\medskip
\noindent Now, with \zimpl it is possible to write this:
{\small
\begin{verbatim}
   var x;
   var y;
   minimize cost: 2 * x + 3 * y;
   subto c1: x + y <= 6;
\end{verbatim}
}
\noindent and have it automatically translated into \mps or \lpf format.
While this looks not much different from what is in the \lpf format,
the difference can be seen, if we use indexed variables.
Here is an example. This is the \lp: 
$$
\begin{array}{rl}
\min& 2 x_1 + 3 x_2 + 1.5 x_3\\
\mbox{subject to}&\sum^3_{i=1} x_i \leq 6\\
&x_i\ge 0\\
\end{array}
$$
And this is how to tell it to \zimpl:

\medskip
\lstset{language=zimpl,%
basicstyle=\sffamily\footnotesize,%
numberstyle=\sffamily\tiny\color{siennabrown},stepnumber=1}
\begin{lstlisting}[frame=]{}
   set I      := { 1 to 3 };
   param c[I] := <1> 2, <2> 3, <3> 1.5;
   var   x[I] >= 0;
   minimize cost: sum <i> in I : c[i] * x[i];
   subto    cons: sum <i> in I : x[i] <= 6;
\end{lstlisting}

\section{Invocation}

In order to run \zimpl on a model given in the file \code{ex1.zpl} type the command:
\begin{verbatim}
   zimpl ex1.zpl
\end{verbatim}
In general terms the command is:
\begin{verbatim}
   zimpl [options] <input-files>
\end{verbatim}
It is possible to give more than one input file. They are read one
after the other as if they were all one big file.
If any error occurs while processing, \zimpl prints out an 
error message and aborts. In case everything goes well, the results 
are written into two or more files, depending on the specified options.

The first output file is the problem generated from the model in either 
\cplex \lp, \mps, or  a ``human readable'' format, 
with extensions \emph{.lp}, \emph{.mps}, or \emph{.hum}, respectively.
The next one is the \emph{table} file, which has the extension \emph{.tbl}.
The table file lists all variable and constraint names used in the model 
and their corresponding names in the problem file. 
The reason for this name translation is the limitation of the length
of names in the \mps format to 
eight characters. Also the \lp format
restricts the length of names. The precise limit is depending on the
version. \cplex~7.0 has a limit of 16 characters, and ignores
silently the rest of the name, while \cplex~9.0 has a limit of 255
characters, but will for some commands only show the first 20 characters 
in the output.

%The third file is and optional \cplex branching order file.

A complete list of all options understood by \zimpl can be found in
Table~\ref{tab:zimpl-options}. 
A typical invocation of \zimpl is for example:
\begin{verbatim}
   zimpl -o solveme -t mps data.zpl model.zpl
\end{verbatim}
This reads the files data.zpl and model.zpl as
input and produces as output the files solveme.mps and solveme.tbl.
Note that in case \mps-output is specified for a maximization problem,
the objective function will be inverted, because the \mps format has no
provision for stating the sense of the objective function. The default
is to assume maximization.

\begin{table}[hbtp]
{\sffamily\small\centering
\begin{tabular}{lp{104mm}}
\toprule
-t \emph{format} & Selects the output format. Can be either \code{lp},
                  which is default, or \code{mps}, or \code{hum}, which
                  is only human readable.\\
-o \emph{name}   & Sets the base-name for the output files.\\ 
                & Defaults to the name of the first input file with 
                  its path and extension stripped off.\\ 
-F \emph{filter} & The output is piped through a filter. A \%s in the
                  string is replaced by the output filename. For example 
                  \code{-F "gzip -c >\%s.gz"} would compress all the 
                  output files.\\
-n \emph{cform}  & Select the format for the generation of constraint
                  names. Can be \code{cm}, which will number them 
                  $1\ldots n$ with a `c' in front. \code{cn} will use 
                  the name supplied in the \code{subto} statement and 
                  number them $1\ldots n$ within the statement. 
                  \code{cf} will use the name given with the \code{subto},
                  then a $1\ldots n$ number like in \code{cm} and then 
                  append all the local variables from the forall statements.\\
-v \emph{0..5}   & Set the verbosity level. 0 is quiet, 1 is default,
                  2 is verbose, 3 and 4 are chatter, and 5 is debug.\\
-D \emph{name=val} & Sets the parameter \emph{name} to the specified
                  value. This is equivalent with having this line in the
                  \zimpl program: \code{param name:=val}.\\
%\hline
-b & Enables bison debug output.\\
-f & Enables flex debug output.\\
-h & Prints a help message.\\
-m & Writes a CPLEX \code{mst} (Mip STart) file.\\
-O & Try to reduce the generated LP by doing some presolve analysis.\\
-r & Writes a CPLEX \code{ord} branching order file.\\
-V & Prints the version number.\\
\bottomrule
\end{tabular}
}
\caption{\zimpl options}
\label{tab:zimpl-options}
\end{table}

% -----------------------------------------------------------------------------
% --- Format
% -----------------------------------------------------------------------------
\section{Format}

Each \zpl-file consists of six types of statements: 
\begin{itemize}
\setlength{\itemsep}{0pt}%
\item Sets
\item Parameters
\item Variables
\item Objective
\item Constraints
\item Function definitions
\end{itemize}
%
Each statement ends with a semicolon. 
Everything from a hash-sign \code{\#}, provided it is not part of a string, to 
the end of the line is treated as a
comment and is ignored.
If a line starts with the word \code{include} followed by a filename in double
quotation marks, then this file is read and processed instead of the line.

%-----------------------------------------------------------------------------------------
\subsection{Expressions}
%-----------------------------------------------------------------------------------------
\zimpl works on its lowest level with two types of data: Strings and
numbers. 
Wherever a number or string is required it is also possible to use a
parameter of the corresponding value type. In most cases expressions are
allowed instead of just a number or a string.
The precedence of operators is the usual one, but
parentheses can always be used to specify the evaluation order explicitly.
%If in doubt use parenthesis.

\subsubsection{Numeric expressions}
A number in \zimpl can be given in the usual format, \eg as 2, -6.5 or 5.234e-12. 
Numeric expressions consist of numbers, numeric valued parameters, and
any of the operators and functions listed in Table~\ref{tab:zimpl-functions}.
Additionally the functions shown in Table~\ref{tab:zimpl-double} can be
used. Note that those functions are only computed with normal double precision
floating-point arithmetic and therefore have limited accuracy.

\begin{table}[hbtp]
\centering
{\sffamily\small
\begin{tabular}{lll}
\toprule
\code{a${}^\wedge$b}, \code{a**b}&$a$ to the power of $b$   & $a^b$\\
\code{a+b}                       &addition                  & $a+b$\\
\code{a-b}                       &subtraction               & $a-b$\\
\code{a*b}                       &multiplication            & $a\cdot b$\\
\code{a/b}                       &division                  & $a/b$\\
\code{a mod b}                   &modulo                    & $a\mod b$\\
%\code{a div b}                   &integer division           & \\
\code{abs(a)}                    &absolute value            & $|a|$\\
\code{sgn(a)}                    &sign                      &
$x>0\Rightarrow 1, x<0\Rightarrow -1,\text{else }0$\\
\code{floor(a)}                  &round down                & $\lfloor a\rfloor$\\
\code{ceil(a)}                   &round up                  & $\lceil a\rceil$\\
\code{a!}                        &factorial                 & $a!$\\
\code{min(S)}                    &minimum of a set          &$\min_{s\in S}$\\
\code{max(S)}                    &maximum of a set          &$\max_{s\in S}$\\
\code{min(a,b,c,\ldots,n)}       &minimum of a list         &$\min (a,b,c,\ldots,n)$\\
\code{max(a,b,c,\ldots,n)}       &maximum of a list         &$\max (a,b,c,\ldots,n)$\\
\code{card(S)}                   &cardinality of a set      &$|S|$\\
\code{ord(A,n,c)}                &ordinal                   &c-th component of the n-th\\ 
                                 &                          & element of set $A$.\\
\code{if a then b}               &                          &\\
\code{else c end}        &\raisebox{1ex}[0cm][0cm]{conditional}
   &\raisebox{1ex}[0cm][0cm]{$\left\{\begin{array}{rl}b,&\text{if }
   a=\text{true}\\c,&\text{if } a=\text{false}\end{array}\right.$}\\\\
\bottomrule
\end{tabular}
}
\caption{Rational arithmetic functions}
\label{tab:zimpl-functions}
\end{table}

%With $\min$ and $\max$ it is possible to find the minimum/maximum
%member of an one dimensional set of numeric values.
%\code{card} gives the cardinality of a set.

\begin{table}[hbtp]
\centering
{\sffamily\small
\begin{tabular}{lll}
\toprule
\code{sqrt(a)}                   &square root               & $\sqrt a$\\
\code{log(a)}                    &logarithm to base 10      & $\log_{10}a$\\
\code{ln(a)}                     &natural logarithm         & $\ln a$\\
\code{exp(a)}                    &exponential function      & $e^a$\\
%\code{random(a,b)}               &random number in $\left[a,b\right]$&\\
\bottomrule
\end{tabular}
}
\caption{Double precision functions}
\label{tab:zimpl-double}
\end{table}

\subsubsection{String expressions}
A string is delimited by double quotation marks \code{"},
\eg \code{"Hallo Keiken"}. 

\subsubsection{Variant expressions}
The following is either a numeric or a string expression, depending on
whether \emph{expression} is a string or a numeric expression:

\smallskip
\code{if} \emph{boolean-expression} \code{then}
\emph{expression} \code{else} \emph{expression} \code{end}

\smallskip
\noindent The same is true for the
\code{ord(}{}\emph{set, tuple-number, component-number}\code{)} function, 
which evaluates to a specific element of a set (details about sets are
covered below). 

\subsubsection{Boolean expressions}
These evaluate either to \emph{true} or to \emph{false}. For numbers and
strings the
relational operators $<$, $<=$, $==$, $!\!\!=$, $>=$, and $>$ are
defined.
Combinations of Boolean expressions with \code{and},
\code{or}, and 
\code{xor}\footnote{$a \text{ xor } b :=a\wedge\neg b\vee \neg a\wedge b$} 
and negation with \code{not} are possible.
The expression \emph{tuple} \code{in} \emph{set-expression} (explained
in the next section) can be used to test set membership of a tuple. 

\subsection{Sets}
Sets consist of tuples. Each tuple can only
be once in a set. The sets in \zimpl are all ordered, but there is no
particular order of the tuples.
Sets are delimited by braces, \code{\{} and~\code{\}},
respectively.
Tuples consist of components. 
The components are either numbers or strings. 
The components are ordered. 
All tuples of a specific set have the same number of components. 
The type of the $n$-th
component for all tuples of a set must be the same, \ie they have to
be either all numbers or all strings. 
The definition of
a tuple is enclosed in angle brackets 
$<$ and $>$, \eg \code{$<$1,2,"x"$>$}. The components are separated by commas.
If tuples are one-dimensional, it is possible 
to omit the tuple delimiters in a list of elements, but in this case  
they must be omitted from all tuples in the definition, \eg
\code{\{1,2,3}\} is valid while \code{\{1,2,$<$3$>$}\} is not.

Sets can be defined with the set statement. It consists of
the keyword \code{set}, the name of the set, an assignment operator
\code{:=} and a valid set expression.

Sets are referenced by the use of a \emph{template} tuple, consisting
of placeholders, which are replaced by the values of the components of
the respective tuple. For example, a set $S$ consisting of two-dimensional
tuples could be referenced by \code{<a,b> in S}. If any of the
placeholders are actual values, only those tuples matching these
values will be extracted.
For example, \code{<1,b> in S} will only get
those tuples whose first component is \code{1}. Please note that if
one of the placeholders is the name of an already defined parameter,
set or variable, it will be substituted. This will result either in an
error or an actual value.

\paragraph{Examples}
{\small 
\begin{verbatim}
set A := { 1, 2, 3 };
set B := { "hi", "ha", "ho" };
set C := { <1,2,"x">, <6,5,"y">, <787,12.6,"oh"> };
\end{verbatim}
}
\noindent For set expressions the functions and
operators given in Table~\ref{tab:zimpl-set-functions} are defined.

An example for the use of the \code{if} \emph{boolean-expression} \code{then}
\emph{set-expression} \code{else} \emph{set-expression} \code{end} can
be found on page~\pageref{sec:indexed-sets} together with the examples for indexed sets.

\paragraph{Examples}
{\small 
\begin{verbatim}
set D := A cross B;
set E := { 6 to 9 } union A without { 2, 3 }; 
set F := { 1 to 9 } * { 10 to 19 } * { "A", "B" };
set G := proj(F, <3,1>); 
# will give: { <"A",1>, <"A",2"> ... <"B",9> }
\end{verbatim}
}

\begin{table}[htbp]
\centering
{\sffamily\small
\begin{tabular}{llp{61mm}}
\toprule
\code{A*B},&\\
\code{A cross B}   &\raisebox{1ex}[0cm][0cm]{cross product} 
                   &\raisebox{1ex}[0cm][0cm]{$\{(x,y)\mid x\in A\wedge y\in B\}$}\\
\code{A+B},&\\
\code{A union B}   &\raisebox{1ex}[0cm][0cm]{union} 
                   &\raisebox{1ex}[0cm][0cm]{$\{x\mid x\in A\vee x\in B\}$}\\
\code{A inter B}   &intersection & $\{x\mid x\in A\wedge x\in B\}$\\
\code{A$\setminus$B, A-B},&\\
\code{A without B} &\raisebox{1ex}[0cm][0cm]{difference} 
                   &\raisebox{1ex}[0cm][0cm]{$\{x\mid x\in A\wedge x\not\in B\}$}\\
\code{A symdiff B} &symmetric difference& 
   $\{x\mid (x\in A\wedge x\not\in B)\vee(x\in B\wedge x\not\in A)\}$\\
\code{\{n\,{..}\,m\}},& generate, &\\
\code{\{n to m \emph{by s}\}}&(default $s = 1$) & 
   \raisebox{1ex}[0cm][0cm]{$\{x\mid x=n + is \leq m, i\in\NN_0, x,n,m,s\in\ZZ\}$}\\
\code{proj(A, t)}& projection & 
   The new set will consist of $n$-tuples, with\\
   &$t=(e_1,\ldots,e_n)$&
   the $i$-th component being the $e_i$-th component of $A$.\\
\code{if a then b}\\
\code{else c end}        &\raisebox{1ex}[0cm][0cm]{conditional}
   &\raisebox{1ex}[0cm][0cm]{$\left\{\begin{array}{rl}b,&\text{if }
   a=\text{true}\\c,&\text{if } a=\text{false}\end{array}\right.$}\\\\
\bottomrule
\end{tabular}
}
\caption{Set related functions}
\label{tab:zimpl-set-functions}
\end{table}

\subsubsection{Conditional sets}
It is possible to restrict a set to tuples that 
satisfy a Boolean expression. The expression given by the \code{with}
clause is evaluated for each tuple in the set and only tuples for
which the expression evaluates to \emph{true} are included in the new set.

\paragraph{Examples}
{\small 
\begin{verbatim}
set F := { <i,j> in Q with i > j and i < 5 };
set A := { "a", "b", "c" };
set B := { 1, 2, 3 };
set V := { <a,2> in A*B with a == "a" or a == "b" };
# will give: { <"a",2>, <"b",2> }
\end{verbatim}
}
\clearpage
\subsubsection{Indexed sets}
\label{sec:indexed-sets}
It is possible to index one set with another set resulting in a set of sets.
Indexed sets are accessed by adding the index of the set in brackets
\code{[} and \code{]}, like \code{S[7]}. 
Table~\ref{tab:zimpl-idxset-fun} lists the available functions.
There are three possibilities how to assign to an indexed set:
\begin{itemize}
\setlength{\itemsep}{0pt}%
\item The assignment expression is a list of comma-separated pairs, 
      consisting of a tuple from the index set and a set expression to assign.
\item If an index tuple is given as part of the index, 
      \eg \verb|<i> in I|, the assignment is evaluated for each value
      of index tuple.
\item By use of a function that returns an indexed set.
\end{itemize}

\subsubsection{Examples}
{\small 
\begin{verbatim}
set I           := { 1..3 };
set A[I]        := <1> {"a","b"}, <2> {"c","e"}, <3> {"f"};
set B[<i> in I] := { 3 * i };
set P[]         := powerset(I);
set J           := indexset(P);
set S[]         := subsets(I, 2);
set K[<i> in I] := if i mod 2 == 0 then { i } else { -i } end;
\end{verbatim}
}

\begin{table}[htbp]
\centering
{\sffamily\small
\begin{tabular}{llp{5cm}}
\toprule
\code{powerset(A)}& generates all subsets of $A$&$\{X\mid X\subseteq A\}$\\
\code{subsets(A,n)}& generates all subsets of $A$\\
                    & with $n$ elements&$\{X\mid X\subseteq A\wedge |X|=n\}$\\
\code{indexset(A)}&the index set of $A$&$\{1\ldots |A|\}$\\
\bottomrule
\end{tabular}
}
\caption{Indexed set functions}
\label{tab:zimpl-idxset-fun}
\end{table}

%------------------------------------------------------------------------------
\subsection{Parameters}
%------------------------------------------------------------------------------
Parameters can be declared with or without an index
set. Without indexing a parameter is just a single value, which is either
a number or a string. For indexed parameters there is one
value for each member of the set. It is possible to declare a
\emph{default} value.

Parameters are declared in the following way: 
The keyword \code{param} is followed by the name of the parameter
optionally followed by the index set. 
Then after the assignment sign comes a list of pairs. The first element of each
pair is a tuple from the index set, while the second element is the value of
the parameter for this index.

\subsubsection{Examples}
{\small 
\begin{verbatim}
set A := { 12 .. 30 };
set C := { <1,2,"x">, <6,5,"y">, <3,7,"z"> };
param q := 5;
param u[A] := <13> 17, <17> 29, <23> 12 default 99;
param w[C] := <1,2,"x"> 1/2, <6,5,"y"> 2/3;
param x[<i> in { 1 .. 8 } with i mod 2 == 0] := 3 * i;
\end{verbatim}
}

Assignments need not to be complete. 
In the example, no value is given for index \code{$<$3,7,"z"$>$} of
parameter \code{w}. This  
is correct as long as it is never referenced.

\subsubsection{Parameter tables}
It is possible to initialize multi-dimensional indexed parameters from
tables. This is especially useful for two-dimensional parameters.
The data is put in a table structure with \code{$|$} signs on each
margin. Then a headline with column indices has to be added, and one index
for each row of the table is needed. The column index has to be
one-dimensional, but the row index can be
multi-dimensional. The complete index for the entry is built by
appending the column index to the row index.
The entries are separated by commas. Any valid expression is
allowed here. As can be seen in the third example below, it is
possible to add a list of entries after the table.

\subsubsection{Examples}
{\small 
\begin{verbatim}
set I := { 1 .. 10 };
set J := { "a", "b", "c", "x", "y", "z" };

param h[I*J] :=   | "a", "c", "x", "z"   |
                |1|  12,  17, 99,     23 |
                |3|   4,   3,-17, 66*5.5 |
                |5| 2/3, -.4,  3, abs(-4)|
                |9|   1,   2,  0,      3 | default -99;

param g[I*I*I] :=     | 1, 2, 3 |
                  |1,3| 0, 0, 1 | 
                  |2,1| 1, 0, 1 |;

param k[I*I] :=  |  7,  8,  9 |
               |4| 89, 67, 55 |
               |5| 12, 13, 14 |, <1,2> 17, <3,4> 99;
\end{verbatim}
}
\noindent The last example is equivalent to:
{\small 
\begin{verbatim}
param k[I*I] := <4,7> 89, <4,8> 67, <4,9> 44, <5,7> 12, 
                <5,8> 13, <5,9> 14, <1,2> 17, <3,4> 99;
\end{verbatim}
}


%------------------------------------------------------------------------------
\subsection{Variables}
%------------------------------------------------------------------------------
Like parameters, variables can be indexed. 
A variable has to be one out of three possible types: 
Continuous (called \emph{real}), binary or integer. The default type is real. 
Variables may have lower and upper bounds. Defaults are
zero as lower and infinity as upper bound. Binary variables are
always bounded between zero and one.
It is possible to compute the value of the lower or upper bounds
depending on the index of the variable (see the last declaration in the
example). Bounds can also be set to \code{infinity} and \code{-infinity}.

\subsubsection{Examples}
{\small 
\begin{verbatim}
var x1;
var x2 binary;
var y[A] real >= 2 <= 18;
var z[<a,b> in C] integer 
      >= a * 10 <= if b <= 3 then p[b] else 10 end; 
\end{verbatim}
}

%\fbox{Remember: if nothing is specified a lower bound of zero is assumed.}

\subsection{Objective}
There must be at most one objective statement in a model. The objective
can be either \code{minimize} or \code{maximize}. Following the
keyword is a name, a colon : and then a linear term expressing the
objective function.

\subsubsection{Example}
{\small 
\begin{verbatim}
minimize cost: 12 * x1 -4.4 * x2 
   + sum <a> in A : u[a] * y[a]
   + sum <a,b,c> in C with a in E and b > 3 : -a/2 * z[a,b,c];
maximize profit: sum <i> in I : c[i] * x[i];
\end{verbatim}
}

%------------------------------------------------------------------------------
\subsection{Constraints}
%------------------------------------------------------------------------------
The general format for a constraint is:

\smallskip
\verb|subto name: term sense term|

\smallskip
\noindent Alternatively it is also possible to define \emph{ranged} constraints, 
which have the form:

\smallskip
\verb|name: expr sense term sense expr|

\smallskip
\noindent \code{name} can be any name starting with a letter. \code{term} is defined
as in the objective. \code{sense} is one of
\code{<=}, \code{>=} and \code{==}. In case of ranged constraints
both senses have to be equal and may not be \code{==}. \code{expr} is any
valid expression that evaluates to a number.
Many constraints can be generated with one statement by the use of the
\code{forall} instruction, as shown below.

\subsubsection{Examples}
{\small 
\begin{verbatim}
subto time:  3 * x1 + 4 * x2 <= 7;
subto space: 50 >= sum <a> in A: 2 * u[a] * y[a] >= 5;
subto weird: forall <a> in A: sum <a,b,c> in C: z[a,b,c]==55;
subto c21: 6*(sum <i> in A: x[i] + sum <j> in B : y[j]) >= 2;
subto c40: x[1] == a[1] + 2 * sum <i> in A do 2*a[i]*x[i]*3+4;
\end{verbatim}
}

%------------------------------------------------------------------------------
\subsection{Details on \emph{sum} and \emph{forall}}
%------------------------------------------------------------------------------
The general forms are:

\smallskip
\code{forall} \emph{index} \code{do} \emph{term}
\quad and\quad 
\code{sum} \emph{index} \code{do} \emph{term}

\smallskip 
\noindent It is possible to nest several forall instructions.
%It should be noted, that a \code{sum}-expression is a \emph{term}
%itself, so it is possible to nest or concatenate them.
The general form of \emph{index} is: 

\smallskip
\emph{tuple} \code{in} \emph{set} \code{with} \emph{boolean-expression}

\smallskip
\noindent It is allowed to write a colon \code{:} instead of \code{do} and a
vertical bar \code{|} instead of \code{with}.
The number of components in the \emph{tuple} and in the
members of the \emph{set} must match. The \code{with} part of an \emph{index} is
optional. The \emph{set} can be any expression giving a set.

\subsubsection{Examples}
{\small 
\begin{verbatim}
forall <i,j> in X cross { 1 to 5 } without { <2,3> } 
   with i > 5 and j < 2 do 
      sum <i,j,k> in X cross { 1 to 3 } cross Z do 
         p[i] * q[j] * w[j,k] >= if i == 2 then 17 else 53;
\end{verbatim}
}
\noindent Note that in the example \emph{i} and \emph{j} are set by the \code{forall}
instruction. So they are fixed in all invocations of \code{sum}.

%------------------------------------------------------------------------------
\subsection{Details on \emph{if} in constraints}
%------------------------------------------------------------------------------
It is possible to put two variants of a constraint into an
\code{if}-statement. The same applies  for \emph{terms}.
A \code{forall} statement inside the result part of an \code{if} is
also possible.

\subsubsection{Examples}
{\small 
\begin{verbatim}
subto c1: forall <i> in I do
  if (i mod 2 == 0) then  3 * x[i] >= 4
                    else -2 * y[i] <= 3 end;

subto c2: sum <i> in I :
     if (i mod 2 == 0) then 3 * x[i] else -2 * y[i] end <= 3;
\end{verbatim}
}


%------------------------------------------------------------------------------
\subsection{Special ordered sets}
%------------------------------------------------------------------------------
\zimpl can be used to specify special ordered sets (\sos) for an
integer program. If a model contains any \sos a \code{sos} file will
be written together with the \code{lp} or \code{mps} file.
The general format of a special ordered set is:

\smallskip
\verb@sos name: [type1|type2] priority expr : term@

\smallskip
\noindent \code{name} can be any name starting with a letter. \sos use
the same namespace as constraints.
\code{term} is defined as in the objective. \code{type1} or
\code{type2} indicate whether a type-1 or type-2 special ordered set
is declared. The priority is optional and equal to the priority
setting for variables.
Many \sos can be generated with one statement by the use of the
\code{forall} instruction, as shown above.

\subsubsection{Examples}
{\small 
\begin{verbatim}
sos s1: type1: 100 * x[1] + 200 * x[2] + 400 * x[3];
sos s2: type2 priority 100 : sum <i> in I: a[i] * x[i];
sos s3: forall <i> in I with i > 2: 
           type1: (100 + i) * x[i] + i * x[i-1];
\end{verbatim}
}

%------------------------------------------------------------------------------
\subsection{Initializing sets and parameters from a file}
%------------------------------------------------------------------------------
It is possible to load the values for a set or a parameter from a
file. The syntax is:

\smallskip
\code{read} \emph{filename} \code{as} \emph{template} 
[\code{skip} \emph{n}] [\code{use} \emph{n}] 
[\code{fs} \emph{s}] [\code{comment} \emph{s}]

\smallskip
\noindent\emph{filename} is the name of the file to read.
\emph{template} is a string with a template for the tuples to
generate. Each input line from the file is split into fields. The
splitting is done according to the following rules:
Whenever a space, tab, comma, semicolon or colon is encountered
a new field is started. Text that is enclosed in double quotes is not
split and the quotes are always removed. When a field is split all space
and tab characters around the splitting point are removed. If the split is
due to a comma, semicolon or colon, each occurrence of these
characters starts a new field. 

\subsubsection{Examples}
All these lines have three fields:
{\small 
\begin{verbatim}
Hallo;12;3    
Moin   7  2
"Hallo, Peter"; "Nice to meet you" 77
,,2
\end{verbatim}
}

\noindent For each component of the tuple, the number of the field
to use for the value is given, followed by either \code{n} if the
field should be interpreted as a number or \code{s} for a string. 
After the template some optional modifiers can be given. The order
does not matter. 
\code{skip} \emph{n} instructs to skip the first
\emph{n} lines of the file. 
\code{use} \emph{n} limits the number of
lines to use to \emph{n}. 
\code{comment} \emph{s} sets a list of characters that start
comments in the file. Each line is ended when any of the comment
characters is found.
When a file is read, empty lines are skipped and not counted for the
\code{use} clause. They are counted for the \code{skip} clause.
 
\subsubsection{Examples}
{\small
\verb|set P := { read "nodes.txt" as "<1s>" };|

\smallskip
\noindent\verb|nodes.txt:|\\
\verb|   Hamburg            | $\rightarrow$ $<$"Hamburg"$>$\\
\verb|   M�nchen            | $\rightarrow$ $<$"M�nchen"$>$\\
\verb|   Berlin             | $\rightarrow$ $<$"Berlin"$>$\\

\smallskip
\noindent\verb|set Q := { read "blabla.txt" as "<1s,5n,2n>" skip 1 use 2 };|

\smallskip
\noindent\verb|blabla.txt:|\\
\verb|   Name;Nr;X;Y;No     | $\rightarrow$ skip\\   
\verb|   Hamburg;12;x;y;7   | $\rightarrow$ $<$"Hamburg",7,12$>$\\
\verb|   Bremen;4;x;y;5     | $\rightarrow$ $<$"Bremen",5,4$>$\\
\verb|   Berlin;2;x;y;8     | $\rightarrow$ skip\\
   
\smallskip
\noindent\verb|param cost[P] := read "cost.txt" as "<1s> 2n" comment "#";|

\smallskip
\noindent\verb|cost.txt:|\\
\verb|   # Name Price       | $\rightarrow$ skip\\
\verb|   Hamburg 1000       | $\rightarrow$ $<$"Hamburg"$>$ 1000\\
\verb|   M�nchen 1200       | $\rightarrow$ $<$"M�nchen"$>$ 1200\\
\verb|   Berlin  1400       | $\rightarrow$ $<$"Berlin"$>$ 1400\\

\smallskip
\noindent\verb|param cost[Q] := read "haha.txt" as "<3s,1n,2n> 4s";|

\smallskip
\noindent\verb|haha.txt:|\\
\verb|      1:2:ab:con1     | $\rightarrow$ $<$"ab",1,2$>$ "con1"\\
\verb|      2:3:bc:con2     | $\rightarrow$ $<$"bc",2,3$>$ "con2"\\
\verb|      4:5:de:con3     | $\rightarrow$ $<$"de",4,5$>$ "con3"\\
}  

\noindent As with table format input, it is possible to add a list of tuples or
parameter entries after a read statement.
\subsubsection{Examples}
{\small 
\begin{verbatim}
set A := { read "test.txt" as "<2n>", <5>, <6> }; 
param winniepoh[X] := 
   read "values.txt" as "<1n,2n> 3n", <1,2> 17, <3,4> 29;
\end{verbatim}
}  

It is also possible to read a single value into a parameter. In this
case, either the file should contain only a single line, or the read
statement should be instructed by means of a \code{use 1} parameter
only to read a single line.

\subsubsection{Examples}
{\small 
\begin{verbatim}
# Read the fourth value in the fifth line
param n := read "huhu.dat" as "4n" skip 4 use 1
\end{verbatim}
}  

%------------------------------------------------------------------------------
\subsection{Function definitions}
%------------------------------------------------------------------------------
It is possible to define functions within \zimpl. The value a
function returns has to be either a number, a string or a set.
The arguments of a function can only be numbers or strings, but within
the function definition it is possible to access all otherwise
declared sets, parameters and variables.

The definition of a function has to start with 
\code{defnumb}, \code{defstrg} or \code{defset}, depending on the
return value.
Then follows the name
of the function and a list of argument names put in parentheses.
Next is an assignment operator \code{:=} and a valid
expression or set expression.

\subsubsection{Examples}
{\small 
\begin{verbatim}
defnumb dist(a,b)  := sqrt(a*a + b*b);
defstrg huehott(a) := if a < 0 then "hue" else "hott" end;
defset  bigger(i)  := { <j> in K with j > i };
\end{verbatim}
}

%------------------------------------------------------------------------------
\subsection{Extended constraints}
%------------------------------------------------------------------------------
\zimpl has the possibility to generate systems of constraints that
mimic conditional constraints. The general syntax is as follows (note
that the \code{else} part is optional):

\smallskip
\centerline{
\code{vif} \emph{boolean-constraint} \code{then} \emph{constraint} 
[ \code{else} \emph{constraint} ] \code{end}}

\smallskip
\noindent where \emph{boolean-constraint} consists of a linear expression involving
variables. All these variables have to be bounded integer or binary
variables. It is not possible to use any continuous variables or integer
variables with infinite bounds in a \emph{boolean-constraint}.
All comparison operators ($<$, $\le$, $==$, $!\!\!=$, $\ge$, $>$) are
allowed. Also combination of several terms with \code{and},
\code{or}, and \code{xor} and negation with \code{not} is possible.
The conditional constraints (those which follow after \code{then} or
\code{else}) may include bounded continuous variables. 
Be aware that using this construct will lead to
the generation of several additional constraints and variables.

\subsubsection{Examples}
{\small 
\begin{verbatim}
var x[I] integer >= 0 <= 20;
subto c1: vif 3 * x[1] + x[2] != 7 
   then sum <i> in I : y[i] <= 17
   else sum <k> in K : z[k] >= 5 end;
subto c2: vif x[1] == 1 and x[2] > 5 then x[3] == 7 end;
subto c3: forall <i> in I with i < max(I) : 
   vif x[i] >= 2 then x[i + 1] <= 4 end;
\end{verbatim}
}

%------------------------------------------------------------------------------
\subsection{Extended functions}
%------------------------------------------------------------------------------
It is possible to use special functions on terms with variables that
will automatically be converted into a system of inequalities. The
arguments of these functions have to be linear terms consisting of
bounded integer or binary variables. At the moment only the function
\code{vabs(t)} that computes the absolute value of the term \code{t}
is implemented, but functions like the minimum or the maximum of two
terms, or the sign of a term can be implemented in a similar manner.
Again, using this construct will lead to
the generation of several additional constraints and variables.

\subsubsection{Examples}
{\small 
\begin{verbatim}
var x[I] integer >= -5 <= 5;
subto c1: vabs(sum <i> in I : x[i]) <= 15;
subto c2: vif vabs(x[1] + x[2]) > 2 then x[3] == 2 end;
\end{verbatim}
}

%------------------------------------------------------------------------------
\subsection{The \emph{do print} and \emph{do check} commands}
%------------------------------------------------------------------------------
The \code{do} command is special.
It has two possible incarnations:
\code{print} and \code{check}. \code{print} will print to the 
standard output stream
whatever numerical, string, Boolean or set expression, or tuple
follows it. This can be used for example to check if a set has the
expected members, or if some computation has the anticipated result.
\code{check} always precedes a Boolean expression. If this
expression does not evaluate to \emph{true}, the program is aborted
with an appropriate error message. This can be used to assert that
specific conditions are met.
It is possible to use a \code{forall} clause before a \code{print}
or \code{check} statement.

\subsubsection{Examples}
{\small 
\begin{verbatim}
set I := { 1..10 };
do print I;
do forall <i> in I with i > 5 do print sqrt(i);
do forall <p> in P do check sum <p,i> in PI : 1 >= 1;
\end{verbatim}
}
\clearpage
\section{Modeling examples}
In this section we show some examples of well-known problems
translated into \zimpl format.

\subsection{The diet problem}
This is the first example in 
\cite[Chapter 1, page 3]{Chvatal1983}.
It is a classic so-called \emph{diet} problem, see for example
\cite{Dantzig1990} about its implications in practice.

Given a set of foods $F$ and a set of nutrients $N$, we have a table
$\pi_{fn}$ of the amount of nutrient $n$ in food $f$. Now $\Pi_n$
defines how much intake of each nutrient is needed. $\Delta_f$
denotes for each food the maximum number of servings acceptable. 
Given prices $c_f$ for each food, we have to find a selection of foods
that obeys the restrictions and has minimal cost. An integer
variable $x_f$ is introduced for each $f\in F$ indicating the number of servings of
food $f$. Integer variables are used, because only complete servings can be
obtained, \ie half an egg is not an option.
The problem may be stated as:
\begin{align*}
\min&\sum_{f\in F}c_{f} x_{f}&&\mbox{subject to}\nonumber\\
\sum_{f\in F} \pi_{fn} x_{f}&\ge\Pi_n&&\fa n\in N\nonumber\\
0\leq x_{f}&\leq\Delta_f&&\fa f\in F\nonumber\\
x_{f}&\in\NNZ&&\fa f\in F
\end{align*}

\noindent This translates into \zimpl as follows:

\medskip
\lstset{language=zimpl,%
basicstyle=\sffamily\footnotesize,%
numberstyle=\sffamily\tiny\color{siennabrown},stepnumber=1}
\begin{lstlisting}[frame=tb]{}
set Food      := { "Oatmeal", "Chicken", "Eggs",
                   "Milk",    "Pie",     "Pork" };
set Nutrients := { "Energy",  "Protein", "Calcium" };
set Attr      := Nutrients + { "Servings", "Price" };

param needed[Nutrients] := 
   <"Energy"> 2000, <"Protein"> 55, <"Calcium"> 800;

param data[Food * Attr] := 
          |"Servings","Energy","Protein","Calcium","Price"|
|"Oatmeal"|        4 ,    110 ,       4 ,       2 ,     3 |
|"Chicken"|        3 ,    205 ,      32 ,      12 ,    24 |
|"Eggs"   |        2 ,    160 ,      13 ,      54 ,    13 |
|"Milk"   |        8 ,    160 ,       8 ,     284 ,     9 |
|"Pie"    |        2 ,    420 ,       4 ,      22 ,    20 |
|"Pork"   |        2 ,    260 ,      14 ,      80 ,    19 |;
#                       (kcal)       (g)      (mg) (cents)       

var x[<f> in Food] integer >= 0 <= data[f, "Servings"];

minimize cost: sum <f> in Food : data[f, "Price"] * x[f];

subto need: forall <n> in Nutrients do
   sum <f> in Food : data[f, n] * x[f] >= needed[n];
\end{lstlisting}

\medskip
\noindent The cheapest meal satisfying all requirements costs 97 cents and
consists of four servings of oatmeal, five servings of milk and two
servings of pie.

%x$Oatmeal                     4.000000
%x$Milk                        5.000000
%x$Pie                         2.000000

\subsection{The traveling salesman problem}
In this example we show how to generate an exponential 
description of the \emph{symmetric traveling salesman problem} (\tsp)
as given for example in 
\cite[Section 58.5]{Schrijver2003}.

Let $G=(V,E)$ be a complete graph, with $V$ being the set of cities
and $E$ being the set of links between the cities. Introducing binary
variables $x_{ij}$ for each $(i,j)\in E$ indicating if edge $(i,j)$ is
part of the tour, the \tsp can be written as:
\begin{align*}
\min &\sum_{(i,j)\in E}d_{ij} x_{ij}&&\text{subject to}\nonumber\\
\sum_{(i,j)\in\delta_v} x_{ij}&=2&&\fa v\in V\nonumber\\
\sum_{(i,j)\in E(U)} x_{ij}&\leq |U|-1&&\fa U\subseteq V, \emptyset\neq
U\neq V\\
x_{ij}&\in\{0,1\}&&\fa (i,j)\in E\nonumber
\end{align*}
%
The data is read in from a file that gives the number of the city and the
x and y coordinates. Distances between cities are assumed Euclidean. For example:
{\footnotesize
\setlength\columnseprule{0.4pt}
\begin{multicols}{2}
\begin{verbatim}
# City       X    Y
Berlin     5251 1340
Frankfurt  5011  864 
Leipzig    5133 1237 
Heidelberg 4941  867 
Karlsruhe  4901  840 
Hamburg    5356  998 
Bayreuth   4993 1159 
Trier      4974  668 
Hannover   5237  972
Stuttgart  4874  909 
Passau     4856 1344 
Augsburg   4833 1089 
Koblenz    5033  759 
Dortmund   5148  741 
Bochum     5145  728 
Duisburg   5142  679 
Wuppertal  5124  715 
Essen      5145  701 
Jena       5093 1158 
\end{verbatim}
\end{multicols}
}

% {\small
% \begin{verbatim}
% #City        x y 
% "Sylt"       1 1
% "Flensburg"  3 1
% "Neum�nster" 2 2
% "Husum"      1 3
% "Schleswig"  3 3
% "Ausacker"   2 4
% \end{verbatim}
% }

\noindent The formulation in \zimpl follows below. Please note that \code{P[]}
holds all subsets of the cities. As a result 
19 cities is about as far as one can get with this approach. 
Information on how to solve much larger instances can be found on the 
\textsc{concorde} website\footnote{http://www.tsp.gatech.edu}.

\medskip
\lstset{language=zimpl,%
basicstyle=\sffamily\footnotesize,%
numberstyle=\sffamily\tiny\color{siennabrown},stepnumber=1}
\begin{lstlisting}[frame=tb]{}
set V             := { read "tsp.dat" as "<1s>" comment "#" };
set E             := { <i,j> in V * V with i < j };
set P[]           := powerset(V);
set K             := indexset(P);

param px[V]       := read "tsp.dat" as "<1s> 2n" comment "#";
param py[V]       := read "tsp.dat" as "<1s> 3n" comment "#";

defnumb dist(a,b) := sqrt((px[a]-px[b])^2 + (py[a]-py[b])^2);

var x[E] binary;

minimize cost: sum <i,j> in E : dist(i,j) * x[i, j];

subto two_connected: forall <v> in V do
   (sum <v,j> in E : x[v,j]) + (sum <i,v> in E : x[i,v]) == 2;

subto no_subtour: 
   forall <k> in K with 
      card(P[k]) > 2 and card(P[k]) < card(V) - 2 do
      sum <i,j> in E with <i> in P[k] and <j> in P[k] : x[i,j] 
      <= card(P[k]) - 1;
\end{lstlisting}

\medskip
\noindent The resulting \lp has 171 variables, 239,925 constraints, and
22,387,149 non-zero entries in the constraint matrix, giving an \mps-file
size of 936\,\textsc{mb}. \cplex solves this to optimality
without branching in less than a minute.\footnote{Only 40
simplex iterations are needed to reach the optimal solution.}

An optimal tour for the data above is 
Berlin, Hamburg, Hannover, Dortmund, Bo\-chum, Wuppertal, Essen,
Duisburg, Trier, Koblenz, Frankfurt, Heidelberg, Karlsruhe, Stuttgart,
Augsburg, Passau, Bayreuth, Jena, Leipzig, Berlin.


\subsection{The capacitated facility location problem}
Here we give a formulation of the \emph{capacitated facility
location} problem. It may also be considered as a kind of \emph{bin packing} problem
with packing costs and variable sized bins, or as a \emph{cutting stock} problem
with cutting costs.

Given a set of possible plants $P$ to build, and a set of stores $S$
with a certain demand $\delta_s$ that has to be satisfied, we have
to decide which plant should serve which store.
We have costs $c_p$ for building plant $p$ and $c_{ps}$
for transporting the goods from plant $p$ to store $s$.
Each plant has only a limited capacity $\kappa_p$.
We insist that each store is served by exactly one plant.
Of course we are looking for the cheapest solution:
%
\begin{align}
\min\sum_{p\in P}c_p z_p& 
   +\!\!\!\sum_{p\in P, s\in S}\!\!\!c_{ps} x_{ps}&&\mbox{subject to}\nonumber\\ 
\sum_{p\in P} x_{ps}& = 1&&\fa s\in S\label{eqn:assign}\\
x_{ps}&\leq z_{p}&&\fa s\in S, p\in P\label{eqn:open}\\
\sum_{s\in S} \delta_s x_{ps}&\leq\kappa_p&&\fa p\in P\label{eqn:capacity}\\
x_{ps},z_p&\in\BB&&\fa p\in P, s\in S\nonumber
\end{align}
%
We use binary variables $z_p$, which are set to one, if and only if plant $p$ is
to be built. Additionally we have binary variables $x_{ps}$, 
which are set to one if and only if plant $p$ serves shop $s$. 
Equation~\eqref{eqn:assign} demands that each store is assigned to
exactly one plant. Inequality~\eqref{eqn:open} makes sure that a plant
that serves a shop is built. Inequality~\eqref{eqn:capacity}
assures that the shops are served by a plant which does not exceed its
capacity. Putting this into \zimpl yields the program shown on the
next page.
The optimal solution for the instance described by the program is to build plants \code{A} and
\code{C}. Stores 2, 3, and 4 are served by plant \code{A} and the others by
plant \code{C}. The total cost is 1457.

\clearpage
\lstset{language=zimpl,%
basicstyle=\sffamily\footnotesize,%
numberstyle=\sffamily\tiny\color{siennabrown},stepnumber=1}
\begin{lstlisting}[frame=]{}
set PLANTS := { "A", "B", "C", "D" };
set STORES := { 1 .. 9 };
set PS     := PLANTS * STORES;

# How much does it cost to build a plant and what capacity 
# will it then have?
param building[PLANTS]:= <"A"> 500, <"B"> 600, <"C"> 700, <"D"> 800;
param capacity[PLANTS]:= <"A">  40, <"B">  55, <"C">  73, <"D">  90;

# The demand of each store
param demand  [STORES]:= <1> 10, <2> 14,
                         <3> 17, <4>  8,
                         <5>  9, <6> 12,
                         <7> 11, <8> 15,
                         <9> 16;

# Transportation cost from each plant to each store
param transport[PS] := 
      |  1,  2,  3,  4,  5,  6,  7,  8,  9 |
  |"A"| 55,  4, 17, 33, 47, 98, 19, 10,  6 |
  |"B"| 42, 12,  4, 23, 16, 78, 47,  9, 82 | 
  |"C"| 17, 34, 65, 25,  7, 67, 45, 13, 54 |
  |"D"| 60,  8, 79, 24, 28, 19, 62, 18, 45 |;

var x[PS]     binary;  # Is plant p supplying store s ?
var z[PLANTS] binary;  # Is plant p built ?

# We want it cheap
minimize cost: sum <p> in PLANTS : building[p] * z[p]
             + sum <p,s> in PS : transport[p,s] * x[p,s];

# Each store is supplied by exactly one plant
subto assign: 
  forall <s> in STORES do 
     sum <p> in PLANTS : x[p,s] == 1;
   
# To be able to supply a store, a plant must be built
subto build: 
   forall <p,s> in PS do 
      x[p,s] <= z[p];

# The plant must be able to meet the demands from all stores
# that are assigned to it
subto limit: 
   forall <p> in PLANTS do
      sum <s> in S : demand[s] * x[p,s] <= capacity[p];
\end{lstlisting}
\clearpage

\subsection{The $n$-queens problem}
\label{ssec:example:n-queens-problem}
The problem is to place $n$ queens on a $n\times n$ chessboard so that no
two queens are on the same row, column or diagonal.
The $n$-queens problem is a classic combinatorial search problem
often used to test the performance of algorithms that solve satisfiability
problems. Note though, that there are algorithms available which need
linear time in practise, like, for example, those of \cite{SosicGu1991}.
We will show four different models for the problem and compare
their performance. 

\subsubsection{The integer model}
The first formulation uses 
one general integer variable for each row of the board.
Each variable can assume the value of a column, \ie we have $n$ variables 
with bounds $1\ldots n$. Next we use the 
\code{vabs} extended function to model an \emph{all different}
constraint on the variables (see constraint c1). 
This makes sure that no queen is located
on the same column than any other queen.
The second constraint (c2) is used to block all the diagonals of a
queen by demanding that the absolute value of the row
distance and the column distance of each pair of queens are
different. We model $a\neq b$ by $\mbox{abs}(a-b)\geq 1$.

Note that this formulation only works if a queen can be placed in each
row, \ie if the size of the board is at least $4\times4$.

\medskip
%\noindent\textsf{Integer formulation}
\lstset{language=zimpl,%
basicstyle=\sffamily\footnotesize,%
numberstyle=\sffamily\tiny\color{siennabrown},stepnumber=1}
\begin{lstlisting}[frame=tb]{}
param queens := 8;

set C := { 1 .. queens };
set P := { <i,j> in C * C with i < j };
 
var x[C] integer >= 1 <= queens;

subto c1: forall <i,j> in P do vabs(x[i] - x[j]) >= 1;
subto c2: forall <i,j> in P do 
             vabs(vabs(x[i] - x[j]) - abs(i - j)) >= 1;
\end{lstlisting}

\medskip
\noindent The following table shows the performance of the model. Since the
problem is modeled as a
pure satisfiability problem, the solution time depends only on how
long it takes to find a feasible solution.\footnote{Which is, in fact,
rather random.} 
The columns titled \emph{Vars}, \emph{Cons}, and \emph{NZ} denote the
number of variables, constraints and non-zero entries in the
constraint matrix of the generated integer program. 
\emph{Nodes} lists the number of branch-and-bound nodes evaluated by
the solver, and \emph{time} gives the solution time in \cpu seconds. 
\begin{center}
{\sffamily\small
\begin{tabular}{crrrrrr}
\toprule
Queens & Vars & Cons &   NZ      & Nodes & Time [s]\\
\midrule
   8    &   344 &   392 &    951 &   1,324   & $<$1\\
  12    &   804 &   924 &  2,243 & 122,394   &  120\\
  16    & 1,456 & 1,680 &  4,079 & $>$1 mill.& $>$1,700\\
\bottomrule
\end{tabular}
}
\end{center}
As we can see, between 12 and 16 queens is the maximum instance size
we can expect to solve with this model. Neither changing the \cplex parameters
to aggressive cut generation nor setting emphasis on integer
feasibility improves the performance significantly.

\subsubsection{The binary models}
Another approach to model the problem is to have 
one binary variable for each square of the
board. The variable is one if and only if a queen is on this square
and we maximize the number of queens on the board.

For each square we compute in advance which other squares are blocked if a queen is
placed on this particular square. Then the extended \code{vif}
constraint is used to set the variables of the blocked squares to zero if a
queen is placed. 

\medskip
%\noindent\textsf{Binary formulation A}
\lstset{language=zimpl,%
basicstyle=\sffamily\footnotesize,escapechar=@,%
numberstyle=\sffamily\tiny\color{siennabrown},stepnumber=1}
\begin{lstlisting}[frame=tb]{}
param columns := 8;

set C   := { 1 .. columns };
set CxC := C * C;

set TABU[<i,j> in CxC] := { <m,n> in CxC with (m != i or n != j)
   and (m == i or n == j or abs(m - i) == abs(n - j)) };

var x[CxC] binary;

maximize queens: sum <i,j> in CxC : x[i,j];

subto c1: forall <i,j> in CxC do vif x[i,j] == 1 then @\label{queens-vif}@
              sum <m,n> in TABU[i,j] : x[m,n] <= 0 end;
\end{lstlisting}

\medskip
\noindent Using extended formulations can make the models more comprehensible.
For example, replacing constraint c1 in line \ref{queens-vif} with
an equivalent one that does not use \code{vif} as shown below,
leads to a formulation that is much harder to understand.

\medskip
\lstset{language=zimpl,%
basicstyle=\sffamily\footnotesize,%
numberstyle=\sffamily\tiny\color{siennabrown},stepnumber=1}
\begin{lstlisting}[firstnumber=13]{}
subto c2: forall <i,j> in CxC do
             card(TABU[i,j]) * x[i,j] 
           + sum <m,n> in TABU[i,j] : x[m,n] <= card(TABU[i,j]);
\end{lstlisting}

\medskip 
\noindent After the application of the \cplex presolve procedure both
formulations result in identical integer programs. The performance of
the model is shown in the following table. \emph{S} indicates the
\cplex settings used: Either \emph{(D)efault},
\emph{(C)uts}\footnote{Cuts: \code{mip cuts all 2} and \code{mip strategy probing 3}.}, or
\emph{(F)easibility}\footnote{Feasibility: \code{mip cuts all -1} and \code{mip emph 1}}.
\emph{Root Node} indicates 
the objective function value of the \lp relaxation of the root node.
\begin{center}
{\sffamily\small
\begin{tabular}{ccrrrrrrr}
\toprule
Queens & S & Vars & Cons &   NZ   & Root Node & Nodes & Time [s]\\
\midrule
   8   & D &   384 &   448 &   2,352 & 13.4301 &     241 & $<$1\\
       & C &       &       &         &  8.0000 &       0 & $<$1\\
  12   & D &   864 & 1,008 &   7,208 & 23.4463 &  20,911 & 4\\
       & C &       &       &         & 12.0000 &       0 & $<$1\\
  16   & D & 1,536 & 1,792 &  16,224 & 35.1807 & 281,030 & 1,662\\
       & C &       &       &         & 16.0000 &      54 & 8\\
  24   & C & 3,456 & 4,032 &  51,856 & 24.0000 &      38 & 42\\
  32   & C & 6,144 & 7,168 & 119,488 & 56.4756 & $>$5,500& $>$2,000\\
\bottomrule
\end{tabular}
}
\end{center}
This approach solves instances with more than 24 queens. 
The use of aggressive cut generation improves the upper
bound on the objective function significantly, though 
it can be observed that for values of
$n$ larger than 24 \cplex is not able to deduce the trivial upper
bound of $n$.\footnote{For the 32 queens instance the optimal solution is found
after 800 nodes, but the upper bound is still 56.1678.}
If we use the following formulation instead of constraint c2, this changes:

\medskip
%\noindent\textsf{Binary formulation B}
\lstset{language=zimpl,%
basicstyle=\sffamily\footnotesize,%
numberstyle=\sffamily\tiny\color{siennabrown},stepnumber=1}
\begin{lstlisting}[firstnumber=13]{}
subto c3: forall <i,j> in CxC do
             forall <m,n> in TABU[i,j] do x[i,j] + x[m,n] <= 1;
\end{lstlisting}

\medskip
\noindent As shown in the table below, the optimal upper bound on the objective
function is always found in the root node. This leads to a similar
situation as in the integer formulation, \ie the solution time depends
mainly on the time it needs to find the optimal solution. While
reducing the number of branch-and-bound nodes evaluated, 
aggressive cut generation increases the total solution time.

With this approach instances up to 96 queens can be solved. 
At this point the integer program gets too large to be generated. 
Even though the \cplex presolve routine is able to aggregate the
constraints  again, \zimpl needs too much memory to generate the \ip.
The column labeled \emph{Pres. NZ} lists the number of non-zero entries
after the presolve procedure.

\begin{center}
{\sffamily\small
\begin{tabular}{ccrrrrrrrr}
\toprule
       &   &      &      &        & Pres. & Root &       & Time    \\
Queens & S & Vars & Cons &   NZ   &  NZ   & Node & Nodes & [s]\\
\midrule
%   8   & D &    64 &     1,456 &     2,912 &    528 &  8.0 &       0 & $<$1\\
  16   & D &   256 &    12,640 &    25,280 &  1,594 & 16.0 &       0 & $<$1\\
  32   & D & 1,024 &   105,152 &   210,304 &  6,060 & 32.0 &      58 & 5  \\
  64   & D & 4,096 &   857,472 & 1,714,944 & 23,970 & 64.0 &     110 & 60\\
  64   & C &       &           &           &        & 64.0 &      30 & 89\\
  96   & D & 9,216 & 2,912,320 & 5,824,640 & 53,829 & 96.0 &      70 & 193\\
  96   & C &       &           &           &        & 96.0 &      30 & 410\\
  96   & F &       &           &           &        & 96.0 &      69 & 66\\
\bottomrule
\end{tabular}
}
\end{center}

Finally, we will try the following set packing formulation:

\medskip
%\noindent\textsf{Set packing formulation}
\lstset{language=zimpl,%
basicstyle=\sffamily\footnotesize,%
numberstyle=\sffamily\tiny\color{siennabrown},stepnumber=1}
\begin{lstlisting}[frame=tb,firstnumber=13]{}
subto row: forall <i> in C do
   sum <i,j> in CxC : x[i,j] <= 1;

subto col: forall <j> in C do
   sum <i,j> in CxC : x[i,j] <= 1;

subto diag_row_do: forall <i> in C do
   sum <m,n> in CxC with m - i == n - 1: x[m,n] <= 1;
      
subto diag_row_up: forall <i> in C do
   sum <m,n> in CxC with m - i == 1 - n: x[m,n] <= 1;
      
subto diag_col_do: forall <j> in C do
   sum <m,n> in CxC with m - 1 == n - j: x[m,n] <= 1;

subto diag_col_up: forall <j> in C do
   sum <m,n> in CxC with card(C) - m == n - j: x[m,n] <= 1;
\end{lstlisting}

\medskip
\noindent Here again, the upper bound on the objective function is always
optimal. The size of the generated \ip is even smaller than that of
the former model after presolve. The results for different instances
size are shown in the following table:
\begin{center}
{\sffamily\small
\begin{tabular}{ccrrrrrrr}
\toprule
Queens & S & Vars & Cons    &   NZ      & Root Node & Nodes & Time [s]\\
\midrule
%   8   & D &     64 &      48 &       272 &   8.0 &     0 & $<$1\\
%  16   & D &    256 &      96 &      1056 &  16.0 &     8 & $<$1\\
%  32   & D &  1,024 &     192 &      4160 &  32.0 &     0 & $<$1\\
  64   & D &  4,096 &     384 &    16,512 &  64.0 &     0 & $<$1\\
  96   & D &  9,216 &     576 &    37,056 &  96.0 &  1680 & 331\\
  96   & C &        &         &           &  96.0 &  1200 & 338\\
  96   & F &        &         &           &  96.0 &   121 &  15\\
 128   & D & 16,384 &     768 &    65,792 & 128.0 & $>$7000 &$>$3600 &\\
 128   & F &        &         &           & 128.0 &   309 &  90\\
\bottomrule
\end{tabular}
}
\end{center}
In case of the 128 queens instance with default settings, a solution with 127 queens is found
after 90 branch-and-bound nodes, but \cplex was not able to find the
optimal solution within an hour. From the performance of the Feasible
setting it can be presumed that generating cuts is not beneficial for
this model.

